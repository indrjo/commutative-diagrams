%=====[ the plan ]==============================================================

% The bapto mechanism is the following composition of procedures
% wired together mostly using TikZ keys as pseudo-callbacks:

%  ,--- node contents ~~~~~~~~~~~bapto~~~~~~~~~~~> name <--,
%  0                                                       4
%  '--> input --1-> process --2-> output --3-> dispatch ---'

%=====[ TiKz magic ]============================================================

% First of all, a gizmo to handle optional console output is defined.

% \newif\ifbaptoisverbose
% \pgfqkeys{/bapto}{
    % verbose/.is if=baptoisverbose,
    % verbose=true,
    % log/.code={
        % \ifbaptoisverbose
            % \let~\space
            % \newlinechar=`\^
            % \message{kD.bapto ~~~~~~ > #1^}
        % \fi
    % },
% }

% (0) gets wired using the "self naming" key on a node.

\pgfqkeys{/tikz}{
    self naming/.style={
        /tikz/node contents/.forward to=/katharizo/input,
        /katharizo/output/.forward to=/bapto/dispatcher,
    },
}

% Finally we define an handler for the dispatcher behaviour
% and then the dispatcher itself.

\newif\ifbaptooverwrites
\pgfqkeys{/bapto}{
    overwrite/.is if=baptooverwrites,
    overwrite=false,
    dispatcher/.code={
        \ifx\tikz@fig@name\empty
            \pgfqkeys{/tikz}{name=#1}
        \else
            \ifbaptooverwrites
                \pgfqkeys{/tikz}{name=#1}
            \fi
        \fi
    },
}
