%=====[ the plan ]==============================================================

% The bapto mechanism is the following composition of procedures
% wired together mostly using TikZ keys as pseudo-callbacks:

%  ,--- node contents ~~~~~~~~~~~bapto~~~~~~~~~~~> name <--,
%  0                                                       4
%  '--> input --1-> process --2-> output --3-> dispatch ---'

%=====[ TiKz magic ]============================================================

% First of all, a gizmo to handle optional console output is defined.

\newif\ifbaptoisverbose
\pgfqkeys{/bapto}{
    verbose/.is if=baptoisverbose,
    verbose=true,
    log/.code={
        \ifbaptoisverbose
            \let~\space
            \newlinechar=`\^
            \message{kD.bapto ~~~~~~ > #1^}
        \fi
    },
}

% (0) gets wired using the "self naming" key on a node.

\pgfqkeys{/tikz}{
    self naming/.style={
        /tikz/node contents/.forward to=/bapto/input,
    },
}

% Three possible pre-processing input expansion behaviours are defined.
% These correspond to three mutually exclusive different wirings for (1).
% By default the no-expansion route is wired.

\pgfqkeys{/bapto/expand}{.is choice,
    once/.style={/bapto/input/.style={/bapto/process/.expand once={##1}}},
    full/.style={/bapto/input/.style={/bapto/process/.expanded={##1}}},
    none/.style={/bapto/input/.style={/bapto/process={##1}}},
    none,
}

% This (2) wires in the beating heart of the mechanism,
% i.e. the Plain TeX sanitizer.

\pgfqkeys{/bapto/process}{.code={
    % TODO: this obviously need a cleanup
    \kDSanitize#1\into tmp\GO
    \pgfkeysalso{/bapto/log={{\Meaning} ==> (\tmp)}}
    \pgfkeysalso{/bapto/output/.expand once=\tmp}
},}


% This is (3). Trivial but necessary for ease of development.

\pgfqkeys{/bapto}{
    output/.forward to=/bapto/dispatcher,
}

% Finally we define an handler for the dispatcher behaviour
% and then the dispatcher itself.

\newif\ifbaptooverwrites
\pgfqkeys{/bapto}{
    overwrite/.is if=baptooverwrites,
    overwrite=false,
    dispatcher/.code={
        \ifx\tikz@fig@name\empty
            \pgfqkeys{/tikz}{name=#1}
        \else
            \ifbaptooverwrites
                \pgfqkeys{/tikz}{name=#1}
            \fi
        \fi
    },
}

%=====[ TeX magic ]=============================================================

% No more TikZ code is needed.  What follows is just the Plain TeX sanitizer.

\def\kDMeanCut#1:->#2\into#3\GO%
    {\kDCSDef{#3}{#2}}

\def\kDMeaningOf#1\into#2\GO{%
    \def\kDFoo{#1}%
    \edef\kDBar{\meaning\kDFoo}%
    \expandafter\kDMeanCut\kDBar\into#2\GO%
}

\def\kDSanitize#1\into#2\GO{%
    \kDMeaningOf#1 \into Meaning\GO% I put an extra space for easy parsing
    \expandafter\edef\csname#2\endcsname% and catch it as the empty exception:
        {\ifx\Meaning\space\else\expandafter\kDSieveString\Meaning\GO\fi}%
}

% MEMO: the \space is the only whitespace going to output (between words)
\def\kDSieveString#1 #2\GO%
    {\kDSieveWord#1\GO\ifx\relax#2\else\space\kDSieveString#2\GO\fi}

\def\kDSieveWord#1#2\GO%
    {\kDSieveCharacter#1\GO\ifx\relax#2\else\kDSieveWord#2\GO\fi}

% Spaces after number to avoid stray \relax
% TODO: rethink of all the dangerous characters cross-format-wise
\def\kDSieveCharacter#1\GO{%
    \ifnum`#1=36 \else% $ dollar
    \ifnum`#1=40 \else% ( left round brace
    \ifnum`#1=41 \else% ) right round brace
    \ifnum`#1=44 \else% , comma
    \ifnum`#1=46 \else% . dot
    \ifnum`#1=58 \else% : colon
    \ifnum`#1=92 \else% \ backslash
    #1\fi\fi\fi\fi\fi\fi\fi%
}
