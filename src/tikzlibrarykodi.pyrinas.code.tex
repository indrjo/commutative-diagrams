% ================================================================= FOUNDATION =

\pgfqkeys{/kD}{
    % wrap node contents with inline math markers
    math node contents/.style={
        /tikz/node contents={$#1$},
    },
    % fallback as parameter to given key after searching for key in /tikz
    % fallback to/.code 2 args={%
        % \let\searchname=\pgfkeyscurrentname%
        % \pgfkeysalso{
            % /tikz/\searchname/.try=#1,
            % #2/.retry/.expand once=\searchname,
        % }%
    % },
    % keys to hold kD styles sorted by kind
    % objects/.unknown/.style={
        % /kD/fallback to={#1}{/kD/math node contents},
    % },
    % arrows/.search also=/tikz,
    % labels/.unknown/.style={
        % /kD/fallback to={#1}{/kD/math node contents},
    % },
    % universal styles
    every object/.style={
        /ektropi/restore,% needed inside matrices
        /ektropi/add=/kD/objects,
        % /kD/objects/cd,
        self naming,
        execute at begin node=$,%
        execute at end node=$,%
        % anchor=base,
    },
    every arrow/.style={
        % /ektropi/restore,% actually not needed
        /ektropi/add=/kD/arrows,
    },
    every label/.style={
        /ektropi/restore,% we're inside the edge
        /ektropi/add=/kD/labels,
        self naming,
        auto,
        inner sep=0.5ex,
        font=\everymath\expandafter{\the\everymath\scriptstyle},
    },
    every layout/.style={
        % /ektropi/restore,% actually not needed
        /ektropi/add=/kD/layouts,
        mitra,
        nodes={/kD/every object}
    },
    % basic arrow styles
    /kD/arrows/.cd,
        crossing over/clearance/.initial=0.5ex,
        crossing over/color/.initial=white,
        crossing over/.style={
        /tikz/preaction={
            -,
            draw=\pgfkeysvalueof{/kD/arrows/crossing over/color},
            line width=\pgfkeysvalueof{/kD/arrows/crossing over/clearance},
        },
        },
        shift/.style={
        transform canvas={
            shift={($(\tikztostart)!#1!-90:(\tikztotarget)-(\tikztostart)$)},
        },
        },
        slide/.style={
        transform canvas={
            shift={($(\tikztostart)!#1!0:(\tikztotarget)-(\tikztostart)$)},
        },
        },
    % basic label styles
    /kD/labels/.cd,
        mid/.style={
            fill=white,
            shape=circle,
            anchor=center,
        },
        </.style={
            near start,
        },
        >/.style={
            near end,
        },
    % basic objects styles
    /kD/objects/.cd,
    % basic lattice styles
    /kD/layouts/.cd,
        rectangular/.style 2 args={
            /tikz/column sep={#1,between origins},
            /tikz/row sep={#2,between origins},
        },
        square/.style={
            /kD/layouts/rectangular={#1}{#1},
        },
        golden/.style={
            /kD/layouts/rectangular={1.618*#1}{#1},
        },
        comb/.style={
            /kD/layouts/rectangular={sqrt(4/3)*#1}{#1},
            /tikz/every odd row/.append style={
                /tikz/xshift=tan(30)*#1,
            },
        },
        comb/.default=4em,
        square/.default=4em,
        golden/.default=4em,
    % arrow styles shortcuts
    /kD/arrows/.cd,
        l>/.style={
            ->,
            bend right,
        },
        r>/.style={
            ->,
            bend left,
        },
        ÷/.style={
            crossing over,
        },
        ÷>/.style={
            ->,
            ÷,
        },
}

% ======================================================== FIRST CHAR HANDLERS =
% a pair of shortcuts for node naming and math content

\pgfqkeys{/handlers}{
    first char syntax=true,
    first char syntax/.cd,
        % the character (/.initial=\kDNamingShortcut,
        the character "/.initial=\kDContentShortcut,
}

% \def\kDPeelRoundParentheses(#1){#1}

% \def\kDNamingShortcut#1{%
    % \pgfkeysalso{
        % /tikz/name/.expand once={\kDPeelRoundParentheses#1},
    % }%
% }

\def\kDPeelDoubleQuotes"#1"{#1}

\def\kDContentShortcut#1{%
    \pgfkeysalso{
        /kD/math node contents/.expand once={\kDPeelDoubleQuotes#1},
    }%
}

% ==================================================================== PARSING =

%% OBSOLETE

% \pgfqkeys{/kD/parse}{
  % bisect/.code args={#1at#2into#3and#4}{%
    % \def\Cut#2##1#2##2#2##3\GO%
      % {\def\cnt{##3}\def\one{#2}\def\two{#2#2}%
       % \ifx\cnt\one\kDCSDef{#3}{##1}\kDCSDef{#4}{}\else%
       % \ifx\cnt\two\kDCSDef{#3}{##2}\kDCSDef{#4}{##1}\else%
       % \errmessage{A bisection crashed.}\fi\fi}%
    % \Cut#2#1#2#2#2\GO%
  % },
  % /tikz/edge node string/.code args={[#1]#2}{%
    % \ifx\relax#1\relax\else\tikzset{edge node={node[/kD/parse/label={#1},/kD/render/label]}}\fi%
    % \ifx\relax#2\relax\else\tikzset{edge node string={#2}}\fi%
  % },
  % arrow/.forward to=/kD/current arrow/.style,
  % label/.forward to=/kD/current label/.style,
  % morphism/.style={
    % /kD/parse/bisect={#1}at{:}into{kDSND}and{kDFST},
    % /kD/parse/arrow/.expand once=\kDSND,
    % /kD/parse/labels/.expand once=\kDFST,
  % },
  % chain/.style={
    % /kD/parse/bisect={#1}at{:}into{kDSND}and{kDFST},
    % /kD/current chain/every arrow/.estyle=\kDSND,
    % /kD/current chain/every label/.estyle=\kDFST,
  % },
  % labels/.code={%
    % \ifx\relax#1\relax\else%
    % \def\doit##1##2\GO{\def\tmp{##1}}\def\sqr{[}\doit#1\GO%
    % \ifx\tmp\sqr%
      % \pgfkeysalso{/kD/current arrow/.append style={edge node string={#1}}}\else%
      % \pgfkeysalso{/kD/current arrow/.append style={edge node string={[#1]}}}\fi%
    % \fi%
  % },
% }

% ================================================================== RENDERING =

% OBSOLETE

% \pgfqkeys{/kD/render}{
  % morphism/.style={
    % /kD/arrows/.cd,
    % /kD/every arrow,
    % /kD/current chain/every arrow,
    % /kD/current arrow,
  % },
  % label/.style={
    % /kD/labels/.cd,
    % /kD/every label,
    % /kD/current chain/every label,
    % /kD/current label,
  % },
% }

%%%%%%%%%%%%%%%%%%%%%%%%%%%%%%%%%%%%%%%%%%%%%%%%%%%%%%%%%%%%%%%%%%%%%%%%%%%%%%%%
%\def\kDChopParse#1|#2|#3\GO%
%  {\kDBalancerInit\pgfdecoratedpathlength
%   \kDBalancerTally#1\to One\GO
%   \kDBalancerTally#2\to Two\GO
%   \kDBalancerTally#3\to Thr\GO
%   \expandafter\kDBalancerGauge\One\to Fst\GO
%   \expandafter\kDBalancerGauge\Thr\to Lst\GO
%   \pgfmathsetmacro\start{\Fst}
%   \pgfmathsetmacro\stop{1-\Lst}}
%
%\tikzset{/kD/arrows/chop/.style={
%  decoration={
%    show path construction,
%    curveto code={
%      \kDFullExpandAfter\kDChopParse{#1}\GO
%      \pgfpathcurvebetweentime{\start}{\stop}
%      {\pgfpointdecoratedinputsegmentfirst}
%      {\pgfpointdecoratedinputsegmentsupporta}
%      {\pgfpointdecoratedinputsegmentsupportb}
%      {\pgfpointdecoratedinputsegmentlast}},
%    lineto code={
%      \kDFullExpandAfter\kDChopParse{#1}\GO
%      \pgfpathcurvebetweentime{\start}{\stop}
%      {\pgfpointdecoratedinputsegmentfirst}
%      {\pgfpointdecoratedinputsegmentfirst}
%      {\pgfpointdecoratedinputsegmentlast}
%      {\pgfpointdecoratedinputsegmentlast}}
%  },decorate},
%  % the following key has to be integrated into the syntax of the main key
%  /kD/arrows/schop/.style={/kD/arrows/chop=#1|*|#1},}
%%%%%%%%%%%%%%%%%%%%%%%%%%%%%%%%%%%%%%%%%%%%%%%%%%%%%%%%%%%%%%%%%%%%%%%%%%%%%%%%

% ============================================================ INTERNAL MACROS =
% The TikZ code is used to define mid level macros.

% \def\kDDoMorphism#1 #2 #3\GO%
  % {\path (#1) edge [/kD/parse/morphism={#2}, /kD/render/morphism] (#3);}
% 
% \let\kDThisSource\relax \let\kDLastSource\relax
% \let\kDThisTarget\relax \let\kDLastTarget\relax
% \def\kDLast{*}
% 
% \def\kDDoMorphismChain#1 #2 #3 #4\GO%
  % {\def\kDSource{#1}\ifx\kDSource\kDLast\let\kDSource\kDLastSource\fi%
   % \def\kDTarget{#3}\ifx\kDTarget\kDLast\let\kDTarget\kDLastTarget\fi%
   % \kDDoMorphism{\kDSource} {#2} {\kDTarget}\GO%
   % \ifx\kDThisSource\relax\let\kDThisSource\kDSource\fi%
   % \ifx\relax#4\relax%
   % \let\kDLastSource\kDThisSource\let\kDThisSource\relax%
   % \let\kDLastTarget\kDTarget%
   % \else\kDDoMorphismChain{#3} #4\GO%
   % \fi}
% 
% \def\kDDoMorphismChainWithoutOptions #1;%
  % {\kDDoMorphismChainWithOptions[] #1;}
% 
% \def\kDDoMorphismChainWithOptions[#1] #2;%
  % {\pgfqkeys{/kD/parse}{chain={#1}}%
   % \kDDoMorphismChain#2 \GO}

%=====[ FRONT END ]=============================================================

% Mid level macros are bundled up into high level macros for the final user.

\tikzset{
  kodi/.code={
    \catcode`\|=12% ConTeXt fix <- TODO: insufficient? investigate

    \def\lay{\matrix[/kD/every layout]}
    \def\obj{\kDOzos[/kD/every object]}

    % \kDCSDefOptional mor\with%
      % \kDDoMorphismChainWithOptions\and%
      % \kDDoMorphismChainWithoutOptions\GO
  },
}





