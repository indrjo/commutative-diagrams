% μανδύας • (mandýas)
%   1. cloak
%   2. mantle
%   3. dolman

% Mandias defines the aesthetics aspect of the user level functionality,
% together with some baseline configuration.

% TODO: clean these up
\usetikzlibrary[calc]
% \usetikzlibrary[decorations.pathreplacing]
% \usetikzlibrary{decorations.pathmorphing}
% \usetikzlibrary[decorations.markings]
% \usetikzlibrary[arrows.meta]

\pgfqkeys{/kD}{
%==[ universal styles ]=========================================================
  every diagram/.append style={},
  every layout/.append style={square},
  every object/.append style={
    % TODO: should every object have this style, or only matrix ones?
    shape=kD baseline centered rectangle
  },
  every arrow/.append style={
    >=stealth,
  },
  every label/.append style={
    auto,
    inner sep=0.5ex,
    font=\everymath\expandafter{\the\everymath\scriptstyle}
  },
%==[ arrow styles ]=============================================================
  /kD/arrows/.cd,
    crossing over/clearance/.initial=0.5ex,
    crossing over/color/.initial=white,
    crossing over/.style={
      /tikz/preaction={
        -,
        draw=\pgfkeysvalueof{/kD/arrows/crossing over/color},
        line width=\pgfkeysvalueof{/kD/arrows/crossing over/clearance},
      },
    },
    shove/.style={
      /tikz/transform canvas={
        /tikz/shift={($(\tikztostart)!#1!-90:(\tikztotarget)-(\tikztostart)$)}
      }
    },
    slide/.style={
      /tikz/transform canvas={
        /tikz/shift={($(\tikztostart)!#1!0:(\tikztotarget)-(\tikztostart)$)}
      }
    },
%==[ label styles ]=============================================================
  /kD/labels/.cd,
    mid/.style={
      /tikz/fill=white,
      /tikz/shape=circle,
      /tikz/anchor=center,
      /tikz/inner sep=.25ex
    },
%=[ objects styles ]============================================================
  /kD/objects/.cd,
    % TODO: reflect on these styles
    % rectangular/.style 2 args={
    %   /tikz/node distance=#2 and #1
    % },
    % square/.style={
    %   /kD/objects/rectangular={#1}{#1}
    % },
    % golden/.style={
    %   /kD/objects/rectangular={#1}{0.618*#1}
    % },
    % % comb/.style={
    % %   /kD/objects/rectangular={#1}{sqrt(3/4)*#1},
    % % },
    % % comb/.default=4em,
    % square/.default=4em,
    % golden/.default=4em,
%=[ lattice styles ]============================================================
  /kD/layouts/.cd,
    rectangular/.style 2 args={
      /tikz/column sep={#1,between origins},
      /tikz/row sep={#2,between origins},
    },
    square/.style={
      /kD/layouts/rectangular={#1}{#1}
    },
    golden/.style={
      /kD/layouts/rectangular={#1}{0.618*#1}
    },
    comb/.code args={#1step #2 angle #3}{
      % TODO: write better option parser
      \if\relax#1\pgfkeys{/kD/layouts/horizontal comb=step #2 angle #3}
      \else\pgfkeys{/kD/layouts/#1comb=step #2 angle #3}\fi
    },
    horizontal comb/.style args={step #1 angle #2}{
      /kD/layouts/rectangular={#1}{tan(#2)*#1*0.5},
      /tikz/every odd row/.append style={/tikz/xshift=(#1)*0.5}
    },
    vertical comb/.style args={step #1 angle #2}{
      /kD/layouts/rectangular={tan(#2)*#1*0.5}{#1},
      /tikz/every odd column/.append style={/tikz/yshift=-1*(#1)*0.5}
    },
    comb/.default=step 4.5em angle 60,
    horizontal comb/.default=step 4.5em angle 60,
    vertical comb/.default=step 4.5em angle 60,
    square/.default=4.5em,
    golden/.default=4.5em,
%=[ diagram styles ]============================================================
  /kD/diagrams/.cd,
    grid/.style 2 args={
      /tikz/x={#1},
      /tikz/y={#2},
      /tikz/on grid
    },
    metric/.style 2 args={
      % {1 and 1}{1} is infty-norm, unit circle is unit square
      % {2}{0.5} is 1-norm, circle of radius 2 is rhombus circumscribing base hexagon
      /tikz/node distance={#1},
      % TODO: implement control to allow for explicit units on single factor specs
      /tikz/above left/.code={\tikz@lib@place@handle@{##1}{south east}{-1}{1}{north west}{#2}},
      /tikz/above right/.code={\tikz@lib@place@handle@{##1}{south west}{1}{1}{north east}{#2}},
      /tikz/below left/.code={\tikz@lib@place@handle@{##1}{north east}{-1}{-1}{south west}{#2}},
      /tikz/below right/.code={\tikz@lib@place@handle@{##1}{north west}{1}{-1}{south east}{#2}}
    },
    rectangular/.style 2 args={
      /kD/every layout/.append style={/kD/layouts/rectangular={#1}{#2}},
      % TODO: how much of the following is necessary in layouts?
      /kD/diagrams/grid={#1}{#2},
      /kD/diagrams/metric={1 and 1}{1}
    },
    square/.style={
      /kD/every layout/.append style={/kD/layouts/square={#1}},
      /kD/diagrams/grid={#1}{#1},
      /kD/diagrams/metric={1 and 1}{1}
    },
    golden/.style={
      /kD/every layout/.append style={/kD/layouts/golden={#1}},
      /kD/diagrams/grid={#1}{0.618*#1},
      /kD/diagrams/metric={1 and 1}{1}
    },
    comb/.code args={#1step #2 angle #3}{
      % TODO: write better option parser
      \if\relax#1\pgfkeys{/kD/diagrams/horizontal comb=step #2 angle #3}
      \else\pgfkeys{/kD/diagrams/#1comb=step #2 angle #3}\fi
    },
    horizontal comb/.style args={step #1 angle #2}{
      /kD/every layout/.append style={/kD/layouts/horizontal comb=step #1 angle #2},
      /kD/diagrams/grid={#1*0.5}{#1*tan(#2)*0.5},
      /kD/diagrams/metric={2}{0.5}
    },
    vertical comb/.style args={step #1 angle #2}{
      /kD/every layout/.append style={/kD/layouts/vertical comb=step #1 angle #2},
      /kD/diagrams/grid={#1*tan(#2)*0.5}{#1*0.5},
      /kD/diagrams/metric={2}{0.5}
    },
    comb/.default=step 4.5em angle 60,
    rectangular/.default={6em}{4em},
    square/.default=4.5em,
    golden/.default=4.5em,
}


%==[ baseline centered rectangle shape ]========================================

% The math formula axis height is recovered and stored as a pgf function.

% NOTE: the LuaTeX version is needed just by ConTeXt
\pgfutil@ifluatex
  \directlua{tex.enableprimitives('kD', {'Umathaxis'})}
  \pgfmathdeclarefunction{kD_math_formula_axis_height}{0}{%
    \begingroup%
      $\relax$% update fontdimens
      % See TeX by Topic §23.5 for details.
      \pgfmathreturn\the\kDUmathaxis\textstyle%
    \endgroup}
\else% if using (pdf)tex
  \pgfmathdeclarefunction{kD_math_formula_axis_height}{0}{%
    \begingroup%
      $\relax$% update fontdimens
      % See TeX by Topic §23.5 for details.
      \pgfmathreturn\the\fontdimen22\textfont2%
    \endgroup}
\fi

\pgfqkeys{/kD/baseline centered rectangle}{
  center raise/.initial=kD_math_formula_axis_height
}

% Then the shape is defined by inheritance.

\pgfdeclareshape{kD baseline centered rectangle} {
  % Inherit the rectangle shape.
  \inheritsavedanchors[from={rectangle}]
  \inheritanchor[from={rectangle}]{base}
  \inheritanchor[from={rectangle}]{north}
  \inheritanchor[from={rectangle}]{south}
  \inheritanchor[from={rectangle}]{base west}
  \inheritanchor[from={rectangle}]{north west}
  \inheritanchor[from={rectangle}]{south west}
  \inheritanchor[from={rectangle}]{base east}
  \inheritanchor[from={rectangle}]{north east}
  \inheritanchor[from={rectangle}]{south east}
  \inheritanchor[from={rectangle}]{mid}
  \inheritanchor[from={rectangle}]{mid west}
  \inheritanchor[from={rectangle}]{mid east}
  \inheritbackgroundpath[from={rectangle}]
  % Redefine west, center and east anchors
  % setting their y coordinates to center raise.
  \anchor{center}{\pgf@anchor@rectangle@center\pgfmathsetlength\pgf@y%
    {\pgfkeysvalueof{/kD/baseline centered rectangle/center raise}}}
  \anchor{west}{\pgf@anchor@rectangle@west\pgfmathsetlength\pgf@y%
    {\pgfkeysvalueof{/kD/baseline centered rectangle/center raise}}}
  \anchor{east}{\pgf@anchor@rectangle@east\pgfmathsetlength\pgf@y%
    {\pgfkeysvalueof{/kD/baseline centered rectangle/center raise}}}
  % Save the original anchors as alternate "real" versions.
  \anchor{real center}{\pgf@anchor@rectangle@center}
  \anchor{real west}{\pgf@anchor@rectangle@west}
  \anchor{real east}{\pgf@anchor@rectangle@east}
  % Redefine the border anchor calculation.
  \anchorborder{%
    % (x,y) = target
    % let tempdima = center raise
    \pgfmathsetlength\pgfutil@tempdima%
      {\pgfkeysvalueof{/kD/baseline centered rectangle/center raise}}%
    % let b = (x,y) = target
    \pgf@xb=\pgf@x%
    \pgf@yb=\pgf@y%
    % let (x,y) = south west
    \southwest%
    % let a = (x,y) = south west
    \pgf@xa=\pgf@x%
    \pgf@ya=\pgf@y%
    % let (x,y) = north east
    \northeast%
    % let (x,y) = (x,y) - a = north east - south west = (width, height)
    \advance\pgf@x by-\pgf@xa%
    \advance\pgf@y by-\pgf@ya%
    % let c = (x,y)/2 = (width, height)/2 = (width/2, height/2)
    \pgf@xc=.5\pgf@x%
    \pgf@yc=.5\pgf@y%
    % let a = a + c = south west + (width/2, height/2) = center
    \advance\pgf@xa by\pgf@xc%
    \advance\pgf@ya by\pgf@yc%
    % if by = target y > 0
    \ifdim\pgf@yb>0pt%
      % let (x,y) = north east
      \northeast%
      % let cy = y = north east y
      \pgf@yc=\pgf@y%
      % let cy = cy - center raise = north east y - center raise
      \advance\pgf@yc by-\pgfutil@tempdima%
    \else%
      % let (x,y) = south west
      \southwest%
      % let cy = y = - south west y
      \pgf@yc=-\pgf@y%
      % let cy = cy + center raise = - south west y + center raise
      \advance\pgf@yc by\pgfutil@tempdima%
    \fi
    \edef\pgf@marshal{%
      % calculate the intersection of the half line from the origin
      \noexpand\pgfpointborderrectangle
      % passing through target
      {\noexpand\pgfqpoint{\the\pgf@xb}{\the\pgf@yb}}
      % and the rectangle centered on the origin
      % whose upper right corner is
      % (width/2, +north east y - center raise) if target y > 0
      % (width/2, -south west y + center raise) if target y < 0
      {\noexpand\pgfqpoint{\the\pgf@xc}{\the\pgf@yc}}%
    }%
    % let (x,y) = the intersection
    \pgf@process{\pgf@marshal}%
    % let x = x + ax = width/2 + center x
    \advance\pgf@x by\pgf@xa%
    % let y = y + tempdima = ±(ne/sw y - center raise) + center raise
    \advance\pgf@y by\pgfutil@tempdima%
    % that is, y = + north east y                    if target y > 0
    %          y = - south west y + 2 * center raise if target y < 0
    %
    % NOTE: in essence, we're just compensating for the redefinition
    % of ne/sw anchors that shifted them by cr below the real center.
    %                ┏━━━━━━┯━━━━━━┓  ╮╮              
    %                ┃      │      ┃  ││ + ney - cr   
    %            ╭╭  ┠──────┼──────┨ ╮│╯               y > 0
    % - swy + cr │╰╭ ┣━━━━━━┿━━━━━━┫ ╯╯ ┈┈┈┈┈┈┈┈┈┈┈┈┈┈┈┈┈┈┈┈┈
    %            ╰ ╰ ┗━━━━━━┷━━━━━━┛                   y < 0
  }%
}%
