\documentclass[12pt]{scrbook}
\usepackage[british]{babel}
% \hyphenation{Fortran hy-phen-ation}

\usepackage{libertine}
\usepackage{libertinust1math}
\usepackage[ttdefault=true]{AnonymousPro}

\usepackage[utf8]{inputenc}
\usepackage[T1]{fontenc}

\setkomafont{disposition}{\rmfamily\scshape}


\usepackage{geometry}

\geometry{
  a4paper,
  portrait,
  marginparwidth=4.25cm,
  marginparsep=.75cm,
  % showframe,
  width=11cm,
  hmarginratio=10:25,
  vmarginratio=20:30,
%  footskip=.5in,
}
%\savegeometry{main}

%\usepackage{multicol}
%\usepackage{tabularx}
%\usepackage{verbatimbox}
% \usepackage{wrapfig}

\usepackage{mathtools}
\usepackage{amsfonts}
\usepackage{amssymb}

\usepackage{float}
% \usepackage{framed}
% \usepackage{lipsum}
% \usepackage{enumitem}
% \usepackage{pdfpages}

% \usepackage{caption}
\usepackage{sidenotes}

% \DeclareCaptionStyle{sidecaption}%
% {font=footnotesize,labelfont=sc}

% \usepackage{csquotes}

\usepackage{graphicx}
% \usepackage[makeroom]{cancel}

\usepackage{etoolbox}

% \usepackage{todonotes}
% \usepackage{marginnote}

% \usepackage[lastpage,user]{zref}
% \usepackage[hidelinks]{hyperref}

\usepackage{xcolor}
\usepackage{tikz}
\usetikzlibrary{kodi}

\usepackage{xparse}

\usepackage[
  headwidth=textwithmarginpar,
  footwidth=textwithmarginpar,
]{scrlayer-scrpage}
%\clearpairofpagestyles

% \lofoot{}
% \cofoot{}
% \rofoot{}
% \lohead{}
% \cohead{}
% \rohead{\thepage}


%\usepackage{multicol}

\usepackage{microtype}

\usepackage{listings}
% http://tex.stackexchange.com/a/134860/82186
% \makeatletter
% \lst@Key{lastline}\relax{\ifnumcomp{#1}{<}{0}{%
  % \let\mylst@file\lst@intname\sbox0{\lstinputlisting{\mylst@file}}%
  % \def\lst@lastline{\the\numexpr#1+\value{lstnumber}-1\relax}}%
  % {\def\lst@lastline{#1\relax}}}
% \makeatother


\usepackage{showexpl}
\makeatletter
\lst@Key{postset}\relax{\def\SX@postset{#1}}
\newcommand\SX@postset{}
\renewcommand*\SX@resultInput{%
  \ifx\SX@graphicname\@empty
    \begingroup
      \MakePercentComment\makeatother\catcode`\^^M=5\relax
      \SX@@preset\SX@preset
      \if@SX@rangeaccept
       \let\SX@tempa=\SX@input
      \else
       \let\SX@tempa=\input
      \fi
      \if\SX@scaled ?%
        \let\SX@tempb=\@firstofone
      \else
        \if\SX@scaled !%
          \def\SX@tempb##1{\resizebox{\SX@width}{!}{##1}}%
        \else
          \def\SX@tempb##1{\scalebox{\SX@scaled}{##1}}%
        \fi
      \fi
      \SX@tempb{\SX@tempa{\SX@codefile}}\SX@postset\par
    \endgroup
  \else
    \expandafter\includegraphics\expandafter[\SX@graphicparam]%
      {\SX@graphicname}%
  \fi
}
\makeatother



\lstdefinelanguage{TikZ}
{morekeywords={for},
sensitive=false,
morecomment=[l]{//},
morecomment=[s]{/*}{*/},
morestring=[b]",
}

\lstset{
  language=[LaTeX]TeX,
  basicstyle=\ttfamily\scriptsize,
  backgroundcolor=\color{black!8},
  % keywordstyle=*\color{blue},
  % identifierstyle=\color{orange}\bfseries,
  % morekeywords={\path},
  moretexcs={
    \starttext,\stoptext,\usemodule,
    \tikzpicture,\endtikzpicture,
    \starttikzpicture,\stoptikzpicture,
    \usetikzlibrary,
    \kodi,\endkodi,
    \startkodi,\stopkodi,
    \lay,\obj,\mor,
    \bye
  },
  texcsstyle=*\textbf,
  % morestring=[b]",
  commentstyle=\itshape,
  frame=none,
}

\lstset{explpreset={
  wide,
  pos=o,
  preset=\centering,
  width=\marginparwidth,
  hsep=\marginparsep,
  rframe={}
}}



\usepackage{lipsum}
\usepackage{caption}
\usepackage{subcaption}

\usepackage{hologo}
\def\ConTeXt{\hologo{ConTeXt}}
\def\koDi{{\scshape koDi}}
\def\TikZ{{\scshape TikZ}}

\begin{document}

%===[ TITLE PAGE ]==============================================================

\thispagestyle{empty}
\noindent
\resizebox{\linewidth}{!}{\scshape \kern-.05ex koDi\kern-.06ex}\\[1em]
\resizebox{\linewidth}{!}{\scshape enchiridion}\par
% \vfill\hfill
% \llap{\resizebox{.2\linewidth}{!}{\scshape v1.0.0}}
\newpage

%===[ FOREWORD ]================================================================


% \begin{adjustwidth}{.45\textwidth}{.45\textwidth-\marginparwidth-\marginparsep}
\noindent\koDi\ is a \TikZ\ library. Its purpose
is drawing commutative diagrams.
It is designed precisely to overcome
the shortcomings of traditional ones.
The syntax is minimalistic and intelligible.\par
\hfill{\itshape Paolo al-Imkānī Brasolin}
% \end{adjustwidth}

\newpage


%===[ PRELIMINARIES ]===========================================================

\section{Preliminaries}

\TikZ\ is the only requirement of \koDi.  This ensures compatibility with
most \TeX\ flavours.  Here are minimal working examples for the main dialects:

\begin{figure}[H]
  \begin{adjustwidth}{0sp}{-\marginparwidth-\marginparsep}
  % \centering
    \begin{subfigure}[]{0.3\linewidth}
      \caption*{\TeX}
      \begin{lstlisting}

\input kodi

\kodi
  % diagram here
\endkodi
\bye
      \end{lstlisting}
    \end{subfigure}
    \hfill
    \begin{subfigure}{0.3\linewidth}
      \caption*{\ConTeXt\ module}
      \begin{lstlisting}

\usemodule[kodi]
\starttext
\startkodi
  % diagram here
\stopkodi
\stoptext
      \end{lstlisting}
    \end{subfigure}
    \hfill
    \begin{subfigure}{0.3\linewidth}
      \caption*{\LaTeX\ package}
      \begin{lstlisting}
\documentclass{article}
\usepackage{kodi}
\begin{document}
\begin{kodi}
  % diagram here
\end{kodi}
\end{document}
      \end{lstlisting}
    \end{subfigure}
    \par
    \begin{subfigure}{0.3\linewidth}
      \caption*{\TeX\ (\TikZ\ library)}
      \begin{lstlisting}

\input tikz
\usetikzlibrary kodi

\tikzpicture[kodi]
  % diagram here
\endtikzpicture
\bye
      \end{lstlisting}
    \end{subfigure}
    \hfill
    \begin{subfigure}{0.3\linewidth}
      \caption*{\ConTeXt\ (\TikZ\ library)}
      \begin{lstlisting}

\usemodule[tikz]
\usetikzlibrary[kodi]
\starttext
\starttikzpicture[kodi]
  % diagram here
\stoptikzpicture
\stoptext
      \end{lstlisting}
    \end{subfigure}
    \hfill
    \begin{subfigure}{0.3\linewidth}
      \caption*{\LaTeX\ (\TikZ\ library)}
      \begin{lstlisting}
\documentclass{article}
\usepackage{tikz}
\usetikzlibrary{kodi}
\begin{document}
\begin{tikzpicture}[kodi]
  % diagram here
\end{tikzpicture}
\end{document}
      \end{lstlisting}
    \end{subfigure}
  \end{adjustwidth}
\end{figure}





% A useful {\sc TikZ} feature exclusive to \LaTeX\ is externalization.
% A small expedient is necessary to use it with {\cs koDi}:
% 
% \blank
% 
% \starttyping
% \documentclass{article}
% \usepackage{tikz}
% \usetikzlibrary{kodi}
% \usetikzlibrary{external}
% \tikzexternalize[prefix=tikzpictures/]
% \begin{document}
% \begin{tikzpicture}[kodi]
% -> insert diagram here
% \end{tikzpicture}
% \end{document}
% \stoptyping





% Questo è un testo di prova \sidenote{Questa è una riflessione amargine, dotata anche di una certa lunghezza.}.

\lipsum[1]

\begin{LTXexample}[
  preset={\tikzpicture[kodi]},
  postset={\endtikzpicture}
]
\lay [golden=2em] { A & B & C \\ D & E & F \\ };
\mor A [bend right, ->] B [bend left, ->] C
  -> D [bend right, ->] E [bend left, ->] F;
\end{LTXexample}

\lipsum[2]

\begin{LTXexample}[linerange={2-3}]
\begin{tikzpicture}[kodi]
\lay { A & B \\};
\mor A -> B;
\end{tikzpicture}
\end{LTXexample}

\lipsum






















% 
% 
% % ##############################################################################
% 
% \startbuffer[example-snake_lemma]%%%%%%%%%%%%%%%%%%%%%%%%%%%%%%%%%%%%%%%%%%%%%%
% 
% \unexpanded\def\coker{{\rm coker}\,}
% 
% \lay[golden]{
         % & \ker a   & \ker b   & \ker c   &   \\
         % & A        & B        & C        & 0 \\
% |(0')| 0 & A'       & B'       & C'       &   \\
         % & \coker a & \coker b & \coker c &   \\
% };
% 
% % horizontal chains
% \mor (ker a) -> (ker b) -> (ker c);
% \mor A f:-> B g:-> C -> 0;
% \mor 0' -> A' f':-> B' g':-> C';
% \mor (coker a) -> (coker b) -> (coker c);
% 
% % vetical chains
% \mor[near start] (ker a) -> A a:-> A' -> (coker a);
% \mor[near start] (ker b) -> B b:-> B' -> (coker b);
% \mor[near start] (ker c) -> C c:-> C' -> (coker c);
% 
% the snake
% \coordinate (tail) at ($(ker b)!9/4!(ker c)$);
% \coordinate (head) at ($(coker b)!9/4!(coker a)$);
% \coordinate (belly) at ($(B)!9/16!(B')$);
% \draw[/kD/arrows/crossing over, ->, rounded corners]
  % (ker c) -- (tail) -- (tail|-belly) -- (belly-|head) -- (head) -- (coker a);
% 
% \stopbuffer%%%%%%%%%%%%%%%%%%%%%%%%%%%%%%%%%%%%%%%%%%%%%%%%%%%%%%%%%%%%%%%%%%%%%
% 
% \startbuffer[example-k4_associahedron]%%%%%%%%%%%%%%%%%%%%%%%%%%%%%%%%%%%%%%%%%%
% \def\id{{\mathbf 1}}
% \catcode`ı=\active \letı\id
% \catcode`×=\active \let×\otimes
% 
% \foreach [count=\n] \o in
    % {((w×x)×y)×x,
     % (w×(x×y))×x,
     % w×((x×y)×x),
     % w×(x×(y×x)),
     % (w×x)×(y×x)}
  % \obj (\n) at ({72*\n:9em}) {\o};
% 
% \mor 1 "a_{w,x,y}×ı_z": -> 2
         % "a_{w,x×y,z}": -> 3
       % "ı_w×a_{x,y,z}": -> 4;
% \mor *   "a_{w×x,y,z}": -> 5
         % "a_{w,x,y×z}": -> *;
% \stopbuffer%%%%%%%%%%%%%%%%%%%%%%%%%%%%%%%%%%%%%%%%%%%%%%%%%%%%%%%%%%%%%%%%%%%%%
% 
% \startbuffer[example-pullback]%%%%%%%%%%%%%%%%%%%%%%%%%%%%%%%%%%%%%%%%%%%%%%%%%%
% \lay[comb]{
  % |(P)| A \times_Z B & B \\
        % A            & Z \\
% };
% 
% \obj at ($(Z)!2!(P)$) {Q};
% 
% \mor[swap] P p_1:-> A f:-> Z;
% \mor       * P_2:-> B g:-> *;
% \mor[swap]:[bend right] Q q_1:-> A;
% \mor      :[bend left]  * q_2:-> B;
% \mor [mid]:[dashed]     *   u:-> P;
% \stopbuffer%%%%%%%%%%%%%%%%%%%%%%%%%%%%%%%%%%%%%%%%%%%%%%%%%%%%%%%%%%%%%%%%%%%%%
% 
% \startbuffer[example-chain]%%%%%%%%%%%%%%%%%%%%%%%%%%%%%%%%%%%%%%%%%%%%%%%%%%%%%
% \pgfkeys{/katharizo/expand=full}
% 
% \foreach [count=\m] \a in {A,B,C}
  % \foreach [count=\n] \i in {n-2,n-1,n,n+1,n+2}
    % \obj at ({5em*\n,-3em*\m}) {\a_{\i}};
% 
% \mor (C_{n+1}) -> (B_{n-1});
% \stopbuffer%%%%%%%%%%%%%%%%%%%%%%%%%%%%%%%%%%%%%%%%%%%%%%%%%%%%%%%%%%%%%%%%%%%%%
% 
% \startbuffer[example-ref]%%%%%%%%%%%%%%%%%%%%%%%%%%%%%%%%%%%%%%%%%%%%%%%%%%%%%%%
% \lay[square]{A&B&C&D\\E&F&G&H\\I&L&M&N\\};
% 
% \mor:                     B -> C;
% 
% % * src = prv first src
% \mor:[draw=none]          B -> C;
% \mor:[bend right, green]  * -> D;
% 
% % * tar = prv last tar
% \mor:[draw=none]          B -> C;
% \mor:[bend right, blue]   A -> *;
% 
% \mor:[draw=none]          B -> C;
% \mor:[bend left, red]     A -> * -> D;
% 
% \mor:                     F -> G;
% 
% % + src = prv last tar
% \mor:[draw=none]          F -> G;
% \mor:[bend right, green]  + -> H;
% 
% % + tar = prv first src
% \mor:[draw=none]          F -> G;
% \mor:[bend right, blue]   E -> +;
% 
% \mor:[draw=none]          F -> G;
% \mor:[bend left, red]     E -> + -> H;
% 
% % so, basically, * is the opposite of +
% \mor:      M -> L;
% \mor:[red] I <- * <- N;
% \stopbuffer%%%%%%%%%%%%%%%%%%%%%%%%%%%%%%%%%%%%%%%%%%%%%%%%%%%%%%%%%%%%%%%%%%%%%
% 
% 
% 
% 
% 
% 
% 
% 
% 


\end{document}
