\TikZ\ is the only dependency of \CoDi.
This ensures compatibility with most\footnote{\CoDi\ builds upon \TikZ, which builds upon  {\ttfamily\small pgf}, which after version 3.1 requires at least \hologo{eTeX} version 2. This is inconsequential except in the unlikely event you're using Knuth's original {\ttfamily\small tex} format.} \TeX\ flavours.
Furthermore, it can be invoked both as a standalone and as a \TikZ\ library.
Below are minimal working examples for the main dialects.

\begin{figure}[H]
  \begin{adjustwidth}{0sp}{-\marginparwidth-\marginparsep}
    \begin{subfigure}{\marginparwidth}
      \caption*{\TeX\ package}
      \begin{lstlisting}[gobble=8]

        \RequirePackage{tikz}%[2013/12/13] % pgf version 3.0.0 required
\ProvidesPackage{commutative-diagrams}[<<DATE>> CoDi <<VERSION>> - Commutative Diagrams for TeX]
\usetikzlibrary{commutative-diagrams}
\let\kodi\kDRammaOpen
\let\endkodi\kDRammaShut
\endinput


        \codi
          % diagram here
        \endcodi
        \bye
      \end{lstlisting}
    \end{subfigure}
    \hfill
    \begin{subfigure}{\marginparwidth}
      \caption*{\ConTeXt\ module}
      \begin{lstlisting}[gobble=8]

        \usemodule
          [commutative-diagrams]
        \starttext
        \startcodi
          % diagram here
        \stopcodi
        \stoptext
      \end{lstlisting}
    \end{subfigure}
    \hfill
    \begin{subfigure}{\marginparwidth}
      \caption*{\LaTeX\ package}
      \begin{lstlisting}[gobble=8]
        \documentclass{article}
        \usepackage
          {commutative-diagrams}
        \begin{document}
        \begin{codi}
          % diagram here
        \end{codi}
        \end{document}
      \end{lstlisting}
    \end{subfigure}
    \par
    \begin{subfigure}{\marginparwidth}
      \caption*{\TeX\ (\TikZ\ library)}
      \begin{lstlisting}[gobble=8]

        \input{tikz}
        \usetikzlibrary
          [commutative-diagrams]

        \tikzpicture[codi]
          % diagram here
        \endtikzpicture
        \bye
      \end{lstlisting}
    \end{subfigure}
    \hfill
    \begin{subfigure}{\marginparwidth}
      \caption*{\ConTeXt\ (\TikZ\ library)}
      \begin{lstlisting}[gobble=8]

        \usemodule[tikz]
        \usetikzlibrary
          [commutative-diagrams]
        \starttext
        \starttikzpicture[codi]
          % diagram here
        \stoptikzpicture
        \stoptext
      \end{lstlisting}
    \end{subfigure}
    \hfill
    \begin{subfigure}{\marginparwidth}
      \caption*{\LaTeX\ (\TikZ\ library)}
      \begin{lstlisting}[gobble=8]
        \documentclass{article}
        \usepackage{tikz}
        \usetikzlibrary
          {commutative-diagrams}
        \begin{document}
        \begin{tikzpicture}[codi]
          % diagram here
        \end{tikzpicture}
        \end{document}
      \end{lstlisting}
    \end{subfigure}
  \end{adjustwidth}
\end{figure}

\begin{marginfigure}[0em]
  % \caption*{\TikZ\ externalization}
  \begin{lstlisting}[gobble=4]
    \documentclass{article}
  
    \usepackage
      {commutative-diagrams}
    % Or, equivalently:
    %\usepackage{tikz}
    %\usetikzlibrary
    %  {commutative-diagrams}
    
    \usetikzlibrary{external}
    \tikzexternalize
      [prefix=tikzpics/]
      
    \begin{document}
    \begin{tikzpicture}[codi]
      % diagram here
    \end{tikzpicture}
    \end{document}
  \end{lstlisting}
\end{marginfigure}
  
A useful \TikZ\ feature exclusive to \LaTeX\ is
\NiceURL
  {externalization}
  {http://texdoc.net/texmf-dist/doc/generic/pgf/pgfmanual.pdf\#page=607}.
It is an effective way to boost processing times by (re)\-compiling figures as
external files only when strictly necessary.

A small expedient is necessary to use it with \CoDi: diagrams must be wrapped in
\lstinline|tikzpicture| environments endowed with the \lstinline|/tikz/codi| key.

On the side is an example saving the pictures in the \lstinline|./tikzpics/| folder
to keep things tidy.

\hfill$\therefore$\hfill\null

Basic knowledge of \TikZ\ is assumed. A plethora of excellent
resources exist, so no crash course on the matter will be improvised
here.
Higher proficiency is not necessary, though recommended:
it will make \CoDi\ a pliable framework instead of a black box.
