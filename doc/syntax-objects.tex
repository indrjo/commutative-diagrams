The first of the two macros that \CoDi\ offers is \lstinline|\obj|.
It is polymorphic and can draw both single objects and layouts.

\begin{lstlisting}[style=metacode]
\obj@opt@ <object options> @/opt@{<math>};(@
  \marginpar{\scriptsize {\color{orange!80!black}Orange fragments} are optional.}@)
\obj@opt@ <layout options> @/opt@{<layout>};
\end{lstlisting}

Layouts are described using the customary \TeX\ tabular syntax.

\begin{lstlisting}[style=metacode]
<layout>         := (@\itshape\underbar{<row> <row separator>}@)(@
  \marginpar{\scriptsize \underbar{Underlined fragments} can repeat one or more times.}@)
<row>            := <cell> (@\itshape\color{orange!80!black}\underbar{<cell separator> <cell>}@)
<row separator>  := \\ @opt@[<length>]@/opt@
<cell>           := @opt@|<object options>| @/opt@<math>
<cell separator> := & @opt@[<length>]@/opt@
\end{lstlisting}

The discretionary options syntax is analogous to standard \TikZ\ nodes and
matrices, respectively.

\begin{lstlisting}[style=metacode]
<object options> := (@\itshape\color{orange!80!black}\underbar{[object keylist]}@) @opt@(<name>) at (<coordinate>)@/opt@
<layout options> := (@\itshape\color{orange!80!black}\underbar{[layout keylist]}@) @opt@(<name>) at (<coordinate>)@/opt@
\end{lstlisting}

\hfill$\therefore$\hfill\null

Nothing of the given syntax is specific to \CoDi.
In fact, \lstinline|\obj| can draw both single objects and layouts
by behaving like the standard \TikZ\ macros
\lstinline|\node| and \lstinline|\matrix| respectively.

Furthermore, layouts content is specified using the common \TeX\
tabular syntax.
The only catch is that row and column separators are always mandatory.

Here is a kitchen sink that includes custom spacing:

\begin{tcblisting}{snippet, trim}
\begin{codi}[square=3em]
\obj {
  A & B &[1em] C \\
  D & E &      F \\[-1em]
  G & H &      I \\
};
\end{codi}
\end{tcblisting}

Here is another one that includes custom options:

\begin{tcblisting}{snippet, trim}
\begin{codi}[square=3em]
\obj [red] {
  A & |[blue]| B & C \\
};
\end{codi}
\end{tcblisting}

A standard feature inherited from \TikZ\ worth a mention
is the ability to name a layout and refer to cells
by their row/column index pairs.

\begin{tcblisting}{snippet, trim}
\begin{codi}[square=3em]
\obj (M) { A & A \\ A & A \\ };
\node [draw=red,  shape=circle, minimum size=2em] at (M-1-2) {};
\node [draw=blue, shape=circle, minimum size=2em] at (M-2-1) {};
\end{codi}
\end{tcblisting}
