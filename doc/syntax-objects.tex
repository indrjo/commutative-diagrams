The first of the two macros that \koDi\ offers is \lstinline|\obj|.
It is polymorphic and can draw both single objects and layouts.

\begin{lstlisting}
\obj@opt@ <object options> @/opt@{<math>};(@
  \marginpar{\scriptsize Orange denotes optional fragments.}@)
\obj@opt@ <layout options> @/opt@{<layout>};
\end{lstlisting}

Layouts are described using the customary \TeX\ tabular syntax.

\begin{lstlisting}
<layout>         := (@\itshape\underbar{<row> <row separator>}@)(@
  \marginpar{\scriptsize Underlined fragments are repeated one or more times.}@)
<row>            := <cell> (@\itshape\color{orange!80!black}\underbar{<cell separator> <cell>}@)
<row separator>  := \\ @opt@[<length>]@/opt@
<cell>           := @opt@|<object options>| @/opt@<math>
<cell separator> := & @opt@[<length>]@/opt@
\end{lstlisting}

The discretionary options syntax is analogous to standard \TikZ\ nodes and
matrices, respectively.

\begin{lstlisting}
<object options> := (@\itshape\color{orange!80!black}\underbar{[object keylist]}@) @opt@(<name>) at (<coordinate>)@/opt@
<layout options> := (@\itshape\color{orange!80!black}\underbar{[layout keylist]}@) @opt@(<name>) at (<coordinate>)@/opt@
\end{lstlisting}

\hfill$\therefore$\hfill\null

% \begin{tcblisting}{kodi snippet}
% \tikzpicture[kodi, overlay, remember picture]
% \obj (it) {A};
% \endtikzpicture
% \end{tcblisting}

% \begin{tcblisting}{kodi snippet}
% \tikzpicture[kodi, overlay, remember picture]
% \obj [below right] at (it) {B};
% \endtikzpicture
% \end{tcblisting}

% \begin{tcblisting}{kodi snippet, listing options={firstline=3,lastline=-1}}
% \tikzpicture[kodi,overlay,remember picture]
% \pgfkeys{/kD/every object/.append style={above right=.25in}}
% \obj [red] (kD) at (current page.south west) {\heartsuit kD};
% \endtikzpicture
% \end{tcblisting}

% \begin{tcblisting}{kodi snippet}
% \tikzpicture[kodi,overlay,remember picture]
% \draw (kD) circle (1.5em);
% \endtikzpicture
% \end{tcblisting}
% \null

\begin{tcblisting}{kodi snippet}
\SmashAndCenter{\tikzpicture[kodi, /kD/diagrams/square=2em]
\obj { A & B &[-1em] C \\
       D & E &       F \\[-1em]
       G & H &       I \\ };
\endtikzpicture}
\end{tcblisting}

Objects are automagically named; the latest homonymous prevails.

\begin{tcblisting}{kodi snippet}
\SmashAndCenter{\tikzpicture[kodi, /kD/diagrams/square=2.125em]
\obj { A & A \\ };
\draw (A) circle (1em);
\endtikzpicture}
\end{tcblisting}

Naming an object avoids its automatic labeling.

\begin{tcblisting}{kodi snippet}
\SmashAndCenter{\tikzpicture[kodi, /kD/diagrams/square=2.125em]
\obj { A & |(A')| A \\ };
\draw [red]   (A)  circle (1em);
\draw [green] (A') circle (1em);
\endtikzpicture}
\end{tcblisting}

Naming a layout lets you refer to objects by row and column.

\begin{tcblisting}{kodi snippet}
\SmashAndCenter{\tikzpicture[kodi, /kD/diagrams/square=2.125em]
\obj (M) { A & A \\ A & A \\ };
\draw [red]   (M-1-2) circle (1em);
\draw [green] (M-2-1) circle (1em);
\endtikzpicture}
\end{tcblisting}

