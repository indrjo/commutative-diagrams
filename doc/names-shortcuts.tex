\begingroup\tcbset{trim/.default={3 and -1}}

Two special labels exist: {\ttfamily *} and {\ttfamily +}.

As a source, {\ttfamily *} evaluates to the head of the previous chain.

\begin{tcblisting}{snippet, trim}
\begin{kodi}
\obj [square=2.5em] { A & B & C & \phantom{D} \\ };
\mor B -> C;
\mor * -> A;
\end{kodi}
\end{tcblisting}

As a target, {\ttfamily *} evaluates to the tail of the previous chain.

\begin{tcblisting}{snippet, trim}
\begin{kodi}
\obj [square=2.5em] { \phantom{A} & B & C & D \\ };
\mor B -> C;
\mor D -> *;
\end{kodi}
\end{tcblisting}

The natural use case for {\ttfamily *} is chain gluing.

\begin{tcblisting}{snippet, trim}
\begin{kodi}
\obj [square=2.5em] { A & B \\ D & C \\ };
\mor A -> B -> C;
\mor * -> D -> *;
\end{kodi}
\end{tcblisting}

As a source, {\ttfamily +} evaluates to the tail of the previous chain.

\begin{tcblisting}{snippet, trim}
\begin{kodi}
\obj [square=2.5em] { \phantom{A} & B & C & D \\ };
\mor B -> C;
\mor + -> D;
\end{kodi}
\end{tcblisting}

As a target, {\ttfamily +} evaluates to the head of the previous chain.

\begin{tcblisting}{snippet, trim}
\begin{kodi}
\obj [square=2.5em] { A & B & C & \phantom{D} \\ };
\mor B -> C;
\mor A -> +;
\end{kodi}
\end{tcblisting}

The natural use case for {\ttfamily +} is chain extension.

\begin{tcblisting}{snippet, trim}
\begin{kodi}
\obj [square=2.5em] { A & B & C & D \\ };
\mor B -> C;
\mor A -> + -> D;
\end{kodi}
\end{tcblisting}

The meanings of {\ttfamily *} and {\ttfamily +} swap on opposite chains.

Chain extension can be obtained using {\ttfamily *}.

\begin{tcblisting}{snippet, trim}
\begin{kodi}
\obj [square=2.5em] { A & B & C & D \\ };
\mor B <- C;
\mor D -> * -> A;
\end{kodi}
\end{tcblisting}

Chain gluing can be obtained using {\ttfamily +}.

\begin{tcblisting}{snippet, trim}
\begin{kodi}
\obj [square=2.5em] { A & B \\ D & C \\ };
\mor A <- B <- C;
\mor + -> D -> +;
\end{kodi}
\end{tcblisting}

\endgroup
