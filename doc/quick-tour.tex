Objects are typeset using the \lstinline!\obj! macro.
%
\begin{tcblisting}{kodi snippet}
\nilstrut\smash{\tikzpicture[baseline=(current bounding box.center),kodi]
\obj {X};
\endtikzpicture}
\end{tcblisting}
%
Almost every diagram is laid along a regular grid,
so the customary tabular syntax of \TeX\ is recognized.
%
\begin{tcblisting}{kodi snippet}
\nilstrut\smash{\tikzpicture[baseline=(current bounding box.center),kodi]
\obj {
  A & B \\
  C & D \\
};
\endtikzpicture}
\end{tcblisting}
%
\koDi\ objects are self-aware and clever enough to name themselves
so you can comfortably refer to them.
%
\begin{tcblisting}{kodi snippet}
\nilstrut\smash{\tikzpicture[baseline=(current bounding box.center),kodi]
\obj {\lim F};
\draw (lim F) circle (4ex);
\endtikzpicture}
\end{tcblisting}
%
Morphisms are typeset using the \lstinline!\mor! macro.
%
\begin{tcblisting}{kodi snippet}
\nilstrut\smash{\tikzpicture[baseline=(current bounding box.center),kodi]
\obj { A & B \\ };
\mor A f:-> B;
\endtikzpicture}
\end{tcblisting}
%
Commutative diagrams exist to study composition and commutation
so naturally \koDi\ allows for the chaining of morphisms and
the gluing of chains.
%
\begin{tcblisting}{kodi snippet}
\nilstrut\smash{\tikzpicture[baseline=(current bounding box.center),kodi]
\obj { A & B \\ C & D \\ };
\mor A -> B -> D;
\mor * -> C -> *;
\endtikzpicture}
\end{tcblisting}
%
These are the only two macros defined by \koDi.

There are more features so please continue reading if this got your attention.
