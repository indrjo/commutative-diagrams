The expansion behaviour of the naming routine can be configured
inside any \CoDi\ scope using the \lstinline!expand! key.

\begin{lstlisting}
/codi/expand = none | once | full
\end{lstlisting}

The three available settings correspond to different degrees of expansion.
A side by side comparison completely illustrates their meanings.

\begin{tcblisting}{snippet, trim}
\begin{codi}
\def\B{Z}
\def\A{\B}
\obj{ |[expand=none]| \A &     % name: A (default)
      |[expand=once]| \A &     % name: B
      |[expand=full]| \A \\ }; % name: Z
\mor A -> B -> Z;
\end{codi}
\end{tcblisting}

\hfill$\therefore$\hfill\null

The default behaviour is to avoid expansion in compliance with the principle
that \emph{names should be predictable from the \emph{literal} code}.
Furthermore, it is seldom wise to liberally expand tokens.

There are circumstances in which it is useful to perform token expansion,
though. A useful application is procedural drawing.

\begin{tcblisting}{snippet, trim}
\begin{codi}
\foreach [count=\r] \l in {A,B,C}
  \foreach [count=\c] \n in {n-1,n,n+1}
    \obj [expand=full] at (3em*\c,-2em*\r) {\l_{\n}};
\mor (A_{n}) -> (B_{n+1}) -> (C_{n}) -> (B_{n-1}) -> (A_{n});
\end{codi}
\end{tcblisting}

In some cases finer control is needed. For instance, full expansion
yields unpractical results when parametrizing macros.

\begin{tcblisting}{snippet, trim}
\begin{codi}
\foreach [count=\c] \m in {\lim,\prod}
  \obj [expand=full] at (4em*\c,0) {\m F};
\mor (protect mathop {relax kern z@ mathgroup
                        symoperators lim}nmlimits@ F)
  -> (DOTSI prodop slimits@ F);
\end{codi}
\end{tcblisting}

This explains why a setting to force a single expansion exists.

\begin{tcblisting}{snippet, trim}
\begin{codi}
\foreach [count=\c] \m in {\lim,\prod}
  \obj [expand=once] at (4em*\c,0) {\m F};
\mor (lim F) -> (prod F);
\end{codi}
\end{tcblisting}
