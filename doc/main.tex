%==[ document class ]===========================================================

\documentclass[12pt]{scrbook}

%==[ languages ]================================================================

\usepackage[british]{babel}
% \hyphenation{Fortran hy-phen-ation}

%==[ fonts, encoding ]==========================================================

\usepackage{libertine}
\usepackage{libertinust1math}
\usepackage[ttdefault=true]{AnonymousPro}

\usepackage[utf8]{inputenc}
\usepackage[T1]{fontenc}

%==[ page geometry ]============================================================

\usepackage{geometry}

\geometry{
  a4paper,
  portrait,
  marginparwidth=4.25cm,
  marginparsep=.75cm,
  %% showframe,
  width=11cm,
  hmarginratio=10:25,
  height=23cm,
  vmarginratio=20:30,
}

\makeatletter
\let\org@Gm@pageframes\Gm@pageframes
\renewcommand*{\Gm@pageframes}{%
  \begingroup
    \color{gray}%
    \org@Gm@pageframes
  \endgroup
}
\makeatother

%==[ page layout ]==============================================================

%\usepackage{multicol}
%\usepackage{tabularx}
\usepackage{booktabs}
%\usepackage{verbatimbox}
%\usepackage{wrapfig}

\usepackage{float}
% \usepackage{framed}
% \usepackage{lipsum}
\usepackage{enumitem}
% \usepackage{pdfpages}

% \usepackage{caption}
\usepackage{sidenotes}

% \DeclareCaptionStyle{sidenote}{font=scriptsize}

% \usepackage{todonotes}
% \usepackage{marginnote}

\usepackage[
  headwidth=textwithmarginpar,
  footwidth=textwithmarginpar,
]{scrlayer-scrpage}
%\clearpairofpagestyles

% \lofoot{}
% \cofoot{}
% \rofoot{}
% \lohead{}
% \cohead{}
% \rohead{\thepage}

%==[ page style ]===============================================================

\setkomafont{disposition}{\rmfamily\scshape}

\usepackage{parskip}

% \usepackage{csquotes}

\usepackage{caption}
\usepackage{subcaption}

\DeclareCaptionStyle{marginfigure}{
  font=footnotesize,
  justification=centering,
}

\captionsetup[subfigure]{
  textfont=footnotesize,
  %% singlelinecheck=off,
  justification=centering
}

\setcounter{secnumdepth}{0}

%==[ commonplace math ]=========================================================

\usepackage{mathtools}
\usepackage{amsfonts}
\usepackage{amssymb}

\DeclareMathOperator{\coker}{coker}

% \def\id{{\mathbf 1}}
% \DeclareUnicodeCharacter{0131}{{\noexpand\mathbf R}}
% \DeclareUnicodeCharacter{00D7}{{\otimes}}
% \catcode`ı=\active \letı\id
% \catcode`×=\active \let×\otimes
% NOTE: needed just for the associahedron example
\usepackage{newunicodechar}
\newunicodechar{ı}{\mathbf 1}
\newunicodechar{×}{\otimes}

%==[ graphics ]=================================================================

\usepackage{graphicx}
\usepackage{xcolor}
% TODO: what did I need soul for again?
\usepackage{color,soul}

\usepackage{tikz}
\usetikzlibrary{decorations.pathreplacing}

%==[ koDi ]=====================================================================

% \usetikzlibrary{kodi}
\usepackage{kodi}
\usetikzlibrary{positioning}

%==[ alternatives to koDi ]=====================================================

% So, yeah, I got *really* tired...
\usepackage{filecontents}

% Copied from Xy sourcecode
\def\Xy{\hbox{\kern-.1em X\kern-.3em\lower.4ex\hbox{Y\kern-.15em}}}

%==[ utilities and custom macros ]==============================================

\usepackage{etoolbox}
\usepackage{xparse}

% \NewDocumentCommand{\SmashAndCenter}{m}{%
% \begingroup\setbox0=\hbox{#1}%
% \newdimen\tmp\tmp=\dimexpr-.5\ht0+.5\dp0\relax%
% \hfill\smash{\raisebox{\the\tmp}{\box0}}\hfill\null\endgroup%
% }

\newsavebox\SmashAndCenterBox
\newdimen\SmashandCenterRaise

% TODO: make overflow on left pages go left
\NewDocumentEnvironment{SmashAndCenter}{}{%
\begin{lrbox}{\SmashAndCenterBox}}{%
\end{lrbox}%
\SmashandCenterRaise=\dimexpr-.5\ht\SmashAndCenterBox+.5\dp\SmashAndCenterBox\relax%
\hfill\smash{\raisebox{\the\SmashandCenterRaise}{\usebox\SmashAndCenterBox}}\hfill\null%
}

\def\eg{e.\,g.}
%% \deg\ie{i.\,e.}

%==[ sourcecode and examples ]==================================================

\usepackage{listings}

% http://tex.stackexchange.com/a/336331/82186
\makeatletter
\lst@Key{lastline}\relax{\ifnumcomp{#1}{<}{0}{%
  \let\mylst@file\lst@intname\sbox0{\lstinputlisting{\mylst@file}}%
  \def\lst@lastline{\the\numexpr#1+\value{lstnumber}-1\relax}}%
  {\def\lst@lastline{#1\relax}}}
\makeatother

\usepackage{showexpl}

\makeatletter
\lst@Key{postset}\relax{\def\SX@postset{#1}}
\newcommand\SX@postset{}
\renewcommand*\SX@resultInput{%
  \ifx\SX@graphicname\@empty
    \begingroup
      \MakePercentComment\makeatother\catcode`\^^M=5\relax
      \SX@@preset\SX@preset
      \if@SX@rangeaccept
       \let\SX@tempa=\SX@input
      \else
       \let\SX@tempa=\input
      \fi
      \if\SX@scaled ?%
        \let\SX@tempb=\@firstofone
      \else
        \if\SX@scaled !%
          \def\SX@tempb##1{\resizebox{\SX@width}{!}{##1}}%
        \else
          \def\SX@tempb##1{\scalebox{\SX@scaled}{##1}}%
        \fi
      \fi
      \SX@tempb{\SX@tempa{\SX@codefile}}\SX@postset\par
    \endgroup
  \else
    \expandafter\includegraphics\expandafter[\SX@graphicparam]%
      {\SX@graphicname}%
  \fi
}
\makeatother

% \lstdefinelanguage{TikZ}{
%   morekeywords={for},
%   sensitive=false,
%   morecomment=[l]{//},
%   morecomment=[s]{/*}{*/},
%   morestring=[b]",
% }

\lstdefinestyle{TeX}{
  language=[LaTeX]TeX,
  basicstyle=\ttfamily\lst@ifdisplaystyle\scriptsize\fi,
  backgroundcolor=\color{teal!10},
  % keywordstyle=*\color{blue},
  % identifierstyle=\color{orange}\bfseries,
  % morekeywords={\path},
  alsoother={@},
  moretexcs={
    \starttext,\stoptext,\usemodule,
    \tikzpicture,\endtikzpicture,
    \tikzexternalize,
    \starttikzpicture,\stoptikzpicture,
    \usetikzlibrary,
    \kodi,\endkodi,
    \startkodi,\stopkodi,
    \lay,\obj,\mor,
    \bye,
    \draw,\foreach,
    \ar,
    \psset,\everypsbox,\ncline,\ncarc,
    \xymatrix,
  },
  texcsstyle=*\bfseries,
  % morestring=[b]",
  commentstyle=\itshape\color{black!60},
  frame=none,
  % extendedchars=false,
  inputencoding=utf8,
  escapeinside={(@}{@)},
  moredelim=**[is][\color{orange!80!black}]{@opt@}{@/opt@},
  moredelim=**[is][\color{blue!80!black}]{@nws@}{@/nws@},
  % moredelim=**[is][\color\underbar]{@rep@}{@/rep@},
  moredelim=**[s][\itshape]{<}{>},
  % NOTE: × and ı are needed just for the associahedron example
  literate={~}{{\textvisiblespace}}1 {XOR}{{$\vert$}}1 {:}{{\textbf{:}}}1 {:=}{{$\equiv$}}1 {×}{{\texttimes}}1 {ı}{{\i}}1,
}

\lstset{style=TeX}
% \lstMakeShortInline[style=TeX]"

% \makeatletter
% \newcount\ublvl\ublvl=0
% \newcount\ubdpt\ubdpt=0
% \newdimen\ubgap\ubgap=.2em
% \def\underbra#1{\underline {\sbox \tw@ {\global\advance\ubdpt1\advance\ublvl1#1}\dp \tw@ \dimexpr\ubgap*(\ubdpt-\ublvl-1)\relax \box \tw@ }\ifnum\ublvl=0\ubdpt=0\fi}
% \makeatother

% \lstset{explpreset={
%   wide,
%   basicstyle=\ttfamily\scriptsize,
%   pos=o,
%   width=\marginparwidth,
%   hsep=\marginparsep,
%   rframe={},
%   preset={\centering\tikzpicture[kodi]},
%   postset={\endtikzpicture}
% }}

\usepackage{tcolorbox}
\tcbuselibrary{listingsutf8}
\tcbset{listing utf8=latin1}

\tcbset{
  smash and center/.style={
    if odd page={
      before lower=\begin{SmashAndCenter},
      after lower=\end{SmashAndCenter}
    }{
      before upper=\begin{SmashAndCenter},
      after upper=\end{SmashAndCenter}
    }},
  trim/.style args={#1 and #2}{
    listing options={
      firstline=#1,
      lastline=#2}},
  trim/.default={2 and -1},
  snippet/.style={
    size=tight,
    colback=white,
    colframe=white,
    if odd page={
      listing side text,
      lefthand width=\textwidth,
      righthand width=\marginparwidth,
      halign lower=center,
    }{
      text side listing,
      righthand width=\textwidth,
      lefthand width=\marginparwidth,
      halign upper=center,
    },
    toggle enlargement,
    grow to right by=\marginparsep+\marginparwidth,
    sidebyside gap=\marginparsep,
    smash and center,
    listing options={}
  },
  commented snippet/.style={
    snippet,
    comment={}, % avoids text compilation (possible tcb bug?)
    if odd page={
      listing side comment,
    }{
      comment side listing,
    },
  },
  gallery/.style={
    size=tight,
    colback=white,
    colframe=white,
    if odd page={
      % listing side text,
      % lefthand width=\textwidth,
      % righthand width=\marginparwidth,
      grow to right by=\marginparsep+\marginparwidth,
    }{
      grow to right by=\marginparsep+\marginparwidth,
      % text side listing,
      % righthand width=\textwidth,
      % lefthand width=\marginparwidth,
      % halign upper=center,
    },
    toggle enlargement,
    % grow to right by=\marginparsep+\marginparwidth,
    text above listing,
    listing options={}
  }
}

% \tcbset{kodi snippet/.style={snippet, trim, smash and center}}

\def\nilstrut{\rule{0sp}{0sp}}

%==[ extra stuff ]==============================================================

\usepackage{soul}

% \usepackage{lipsum}

\usepackage{hologo}
\def\ConTeXt{\hologo{ConTeXt}}
\def\koDi{{\scshape koDi}}
\def\TikZ{{\scshape TikZ}}

% \usepackage[lastpage,user]{zref}
\usepackage[
  % colorlinks=true,
  % urlcolor=blue,
  % linkbordercolor=black,
  % pdfborderstyle={/S/U/W .4}% border style will be underline of width 1pt
  % urlbordercolor=cyan        % color of external links
  hidelinks
]{hyperref}

\def\NiceURL#1#2{\href{#2}{\color{blue}\ul{#1}}}

%==[ microtype ]================================================================

\usepackage{microtype}


\begin{document}

%==[ TITLE PAGE ]===============================================================

\thispagestyle{empty}
\noindent
\resizebox{\linewidth}{!}{\scshape koDi}\\[0.62em]
\resizebox{\linewidth}{!}{\scshape kommutative Diagramme für \TeX}\\[1.62em]
\resizebox{\linewidth}{!}{\scshape enchiridion}\par
\vfill
\marginpar{
  \resizebox{\linewidth}{!}{\scshape <<VERSION>>}\\[0.62em]
  \resizebox{\linewidth}{!}{\scshape \today}
}

%==[ FOREWORD ]=================================================================

\newpage
\begin{adjustwidth}{.47\textwidth-\marginparwidth-\marginparsep}{.47\textwidth}
\noindent\koDi\ is a \TikZ\ library. Its aim is\linebreak
making commutative diagrams\linebreak
easy to design, parse and tweak.\par
\end{adjustwidth}

\newpage
\section{Preliminaries}
\TikZ\ is the only requirement of \koDi.  This ensures compatibility with
most \TeX\ flavours.  Here are minimal working examples for the main dialects:

\begin{figure}[H]
  \begin{adjustwidth}{0sp}{-\marginparwidth-\marginparsep}
    \begin{subfigure}{\marginparwidth}
      \caption*{\TeX}
      \begin{lstlisting}

\input kodi

\kodi
  % diagram here
\endkodi
\bye
      \end{lstlisting}
    \end{subfigure}
    \hfill
    \begin{subfigure}{\marginparwidth}
      \caption*{\ConTeXt\ module}
      \begin{lstlisting}

\usemodule[kodi]
\starttext
\startkodi
  % diagram here
\stopkodi
\stoptext
      \end{lstlisting}
    \end{subfigure}
    \hfill
    \begin{subfigure}{\marginparwidth}
      \caption*{\LaTeX\ package}
      \begin{lstlisting}
\documentclass{article}
\usepackage{kodi}
\begin{document}
\begin{kodi}
  % diagram here
\end{kodi}
\end{document}
      \end{lstlisting}
    \end{subfigure}
    \par
    \begin{subfigure}{\marginparwidth}
      \caption*{\TeX\ (\TikZ\ library)}
      \begin{lstlisting}

\input tikz
\usetikzlibrary kodi

\tikzpicture[kodi]
  % diagram here
\endtikzpicture
\bye
      \end{lstlisting}
    \end{subfigure}
    \hfill
    \begin{subfigure}{\marginparwidth}
      \caption*{\ConTeXt\ (\TikZ\ library)}
      \begin{lstlisting}

\usemodule[tikz]
\usetikzlibrary[kodi]
\starttext
\starttikzpicture[kodi]
  % diagram here
\stoptikzpicture
\stoptext
      \end{lstlisting}
    \end{subfigure}
    \hfill
    \begin{subfigure}{\marginparwidth}
      \caption*{\LaTeX\ (\TikZ\ library)}
      \begin{lstlisting}
\documentclass{article}
\usepackage{tikz}
\usetikzlibrary{kodi}
\begin{document}
\begin{tikzpicture}[kodi]
  % diagram here
\end{tikzpicture}
\end{document}
      \end{lstlisting}
    \end{subfigure}
  \end{adjustwidth}
\end{figure}

A useful \TikZ\ feature exclusive to \LaTeX\ is externalization.
A small expedient is necessary to use it with \koDi.
\begin{marginfigure}[-2em]
  \caption*{\TikZ\ externalization}
  \begin{lstlisting}
\documentclass{article}
\usepackage{tikz}
\usetikzlibrary{kodi}
\usetikzlibrary{external}
\tikzexternalize
  [prefix=tikzpicfolder/]
\begin{document}
\begin{tikzpicture}[kodi]
  % diagram here
\end{tikzpicture}
\end{document}
  \end{lstlisting}
\end{marginfigure}

\newpage
\section{Quick tour}
Objects are typeset using the \lstinline|\obj| macro.

\begin{tcblisting}{snippet, trim}
\begin{codi}
\obj {X};
\end{codi}
\end{tcblisting}

Almost every diagram is laid along a regular grid,
so the customary tabular syntax of \TeX\ is recognized.

\begin{tcblisting}{snippet, trim}
\begin{codi}
\obj {
  A & B \\
  C & D \\
};
\end{codi}
\end{tcblisting}

\CoDi\ objects are self-aware and clever enough to name themselves
so you can comfortably refer to them.

\begin{tcblisting}{snippet, trim}
\begin{codi}
\obj {\lim F};
\draw (lim F) circle (4ex);
\end{codi}
\end{tcblisting}

Morphisms are typeset using the \lstinline!\mor! macro.

\begin{tcblisting}{snippet, trim}
\begin{codi}
\obj { A & B \\ };
\mor A f:-> B;
\end{codi}
\end{tcblisting}

Commutative diagrams exist to illustrate composition and commutation,
so \CoDi\ allows arrow chaining and chain gluing.

\begin{tcblisting}{snippet, trim}
\begin{codi}
\obj { A & B \\ C & D \\ };
\mor A -> B -> D;
\mor * -> C -> *;
\end{codi}
\end{tcblisting}

These are the only two macros defined by \CoDi.

There are more features, though.\\
Read on if this caught your attention.

\newpage
\section{Alternatives}
It is only fair to mutely offer a comparison with mainstream packages,
showing idiomatic code to draw the same diagram.

Let \NiceURL
  {\Xy-pic}
  {http://texdoc.net/texmf-dist/doc/generic/xypic/xyrefer.pdf\#page=1}
set the bar with a \emph{verbatim} extract from its manual.

\begin{tcblisting}{snippet}
\xymatrix{
 U \ar@/_/[ddr]_y \ar[dr] \ar@/^/[drr]^x \\
  & X \times_Z Y \ar[d]^q \ar[r]_p
                 & X \ar[d]_f            \\
  & Y \ar[r]^g & Z                       }
\end{tcblisting}

Here is an example adapted from \NiceURL
  {\ttfamily\small pst-node}
  {http://texdoc.net/texmf-dist/doc/generic/pst-node/pst-node-doc.pdf\#page=23}'s
documentation.

\begin{tcblisting}{snippet}
$ \psset{colsep=2.5em, rowsep=2em}
 \begin{psmatrix}
  U \\
 & X\times_Z Y & X \\
 & Y & Z
 \psset{arrows=->, nodesep=3pt}
 \everypsbox{\scriptstyle}
 \ncline{1,1}{2,2}
 \ncarc[arcangle=-10]{1,1}{3,2}_{y}
 \ncarc[arcangle=10]{1,1}{2,3}^{x}
 \ncline{2,2}{3,2}>{q}
 \ncline{2,2}{2,3}_{p}
 \ncline{2,3}{3,3}<{f}
 \ncline{3,2}{3,3}^{g}
 \end{psmatrix}$
\end{tcblisting}

Next one is refitted from the guide to \NiceURL
  {\ttfamily\small\{tikz-cd\}}
  {http://texdoc.net/texmf-dist/doc/latex/tikz-cd/tikz-cd-doc.pdf\#page=3}.

\begin{tcblisting}{snippet}
\begin{tikzcd}[column sep=scriptsize, row sep=scriptsize]
  U
  \arrow[drr, bend left=10, "x"]
  \arrow[ddr, bend right=10, swap, "y"]
  \arrow[dr] & & \\
    & X \times_Z Y \arrow[r, swap, "p"] \arrow[d, "q"]
      & X \arrow[d, swap, "f"] \\
    & Y \arrow[r, "g"]
      & Z
\end{tikzcd}
\end{tcblisting}

Finally, \textbf{\koDi}.

\begin{tcblisting}{snippet}
\begin{kodi}[golden]
  \obj { |(pb)| X \times_Z Y & X \\
                           Y & Z \\ };
  \obj [above left=of pb] {U};

  \mor[swap] pb p:-> X f:-> Z;
  \mor        * q:-> Y g:-> *;

  \mor                       U   -> pb;
  \mor      :[bend left=10]  * x:-> X;
  \mor[swap]:[bend right=10] * y:-> Y;
\end{kodi}
\end{tcblisting}

\newpage
\section{Syntax: objects}
The first of the two macros that \CoDi\ offers is \lstinline|\obj|.
It is polymorphic and can draw both single objects and layouts.

\begin{lstlisting}
\obj@opt@ <object options> @/opt@{<math>};(@
  \marginpar{\scriptsize {\color{orange!80!black}Orange fragments} are optional.}@)
\obj@opt@ <layout options> @/opt@{<layout>};
\end{lstlisting}

Layouts are described using the customary \TeX\ tabular syntax.

\begin{lstlisting}
<layout>         := (@\itshape\underbar{<row> <row separator>}@)(@
  \marginpar{\scriptsize \underbar{Underlined fragments} can repeat one or more times.}@)
<row>            := <cell> (@\itshape\color{orange!80!black}\underbar{<cell separator> <cell>}@)
<row separator>  := \\ @opt@[<length>]@/opt@
<cell>           := @opt@|<object options>| @/opt@<math>
<cell separator> := & @opt@[<length>]@/opt@
\end{lstlisting}

The discretionary options syntax is analogous to standard \TikZ\ nodes and
matrices, respectively.

\begin{lstlisting}
<object options> := (@\itshape\color{orange!80!black}\underbar{[object keylist]}@) @opt@(<name>) at (<coordinate>)@/opt@
<layout options> := (@\itshape\color{orange!80!black}\underbar{[layout keylist]}@) @opt@(<name>) at (<coordinate>)@/opt@
\end{lstlisting}

\hfill$\therefore$\hfill\null

Nothing of the given syntax is specific to \CoDi.
In fact, \lstinline|\obj| can draw both single objects and layouts
by behaving like the standard \TikZ\ macros
\lstinline|\node| and \lstinline|\matrix| respectively.

Furthermore, layouts content is specified using the common \TeX\
tabular syntax.
The only catch is that row and column separators are always mandatory.

Here is a kitchen sink that includes custom spacing:

\begin{tcblisting}{snippet, trim}
\begin{codi}[square=3em]
\obj {
  A & B &[1em] C \\
  D & E &      F \\[-1em]
  G & H &      I \\
};
\end{codi}
\end{tcblisting}

Here is another one that includes custom options:

\begin{tcblisting}{snippet, trim}
\begin{codi}[square=3em]
\obj [red] {
  A & |[blue]| B & C \\
};
\end{codi}
\end{tcblisting}

A standard feature inherited from \TikZ\ worth a mention
is the ability to name a layout and refer to cells
by their row/column index pairs.

\begin{tcblisting}{snippet, trim}
\begin{codi}[square=3em]
\obj (M) { A & A \\ A & A \\ };
\node [draw=red,  shape=circle, minimum size=2em] at (M-1-2) {};
\node [draw=blue, shape=circle, minimum size=2em] at (M-2-1) {};
\end{codi}
\end{tcblisting}

\newpage
\section{Syntax: morphisms}
\begingroup\tcbset{trim/.default={3 and -1}}

The second and last macro that \koDi\ offers is \lstinline|\mor|.
It can draw single or chained morphisms.

\begin{lstlisting}
\mor@opt@ <chain options> @/opt@<object>(@
  \itshape\underbar{\textvisiblespace<morphism>\textvisiblespace<object>}@);(@
  \marginpar{\scriptsize Whitespace marked as \textvisiblespace\ is mandatory.}@)
\end{lstlisting}

Source and target objects are referred to by their name.
  
\begin{lstlisting}
<object>   := @nws@(<name>)@/nws@ (@
  \marginpar{\scriptsize {\color{blue!80!black}Blue fragments} can be either enclosed in the shown delimiters, or a \TeX\ group (not idiomatic), or simply devoid of whitespace.}@)
\end{lstlisting}

Morphisms consist of one or more optional labels and an arrow.
  
\begin{lstlisting}
<morphism> := @opt@<labels> : @/opt@<arrow>
<labels>   := @nws@"<math>"@/nws@ XOR (@\underbar{[{\itshape "<math>", <label keylist>}]}@) (@
  \marginpar{\scriptsize Alternatives are separated by $\vert$s.}@)
<arrow>    := @nws@[<arrow keylist>]@/nws@
\end{lstlisting}
% <labels>   := @nws@"<math>"@/nws@ XOR @nws@[<label keylist>]@/nws@ XOR (@\underbar{[{\itshape <label keylist>}]}@)

Global options can be given to both labels and arrows.

\begin{lstlisting}
<chain options> := [<label keylist>] @opt@: [<arrow keylist>]@/opt@
\end{lstlisting}

\hfill$\therefore$\hfill\null

These rules allow for a label syntax that sprouts gracefully
from the simplest to the most complex case.

\begin{tcblisting}{snippet, trim}
\begin{kodi}
\obj { A & B & C \\ F & E & D \\ };
\mor A -> B;
\mor B f:-> C;
\mor C \hat g:-> D;
\mor D "h i":-> E;
\mor E ["L", above]:-> F;
\mor F ["m", near start]["n", swap]["o", near end]:-> A;
\end{kodi}
\end{tcblisting}

The same holds for arrow syntax.

\begin{tcblisting}{snippet, trim}
\begin{kodi}
\obj { A & B & C \\ };
\mor A -> B;
\mor B [>-, dashed] C;
\end{kodi}
\end{tcblisting}

Global options can be used to minimize local ones
and keep the code terse and readable.

\begin{tcblisting}{snippet, trim}
\begin{kodi}
\obj { A & B & C \\ F & E & D \\ };
\mor [swap]:[bend left] B f:-> C g:>-> D h:>- E i:- B;
\mor :[bend right] E x:-> F y:>-> A z:>- B;
\mor [mid] B m:-> D;
\end{kodi}
\end{tcblisting}

\endgroup

\newpage
\section{Names}
As you'll have guessed by now, objects name themselves.

The process happens in three steps:
\begin{itemize}[nosep]
  \item expand tokens;
  \item replace characters;
  \item apply name, overwriting if necessary.
  \end{itemize}

Each one can be configured in any \koDi\ scope with the keys.

You can control the degree of expansion; the default behaviour (no expansion)
shields you from problems with unsafe macros.


\newpage
\subsection{Names: shortcuts}
\begingroup\tcbset{trim/.default={3 and -1}}

Two special labels exist: {\ttfamily *} and {\ttfamily +}.

As a source, {\ttfamily *} evaluates to the head of the previous chain.

\begin{tcblisting}{snippet, trim}
\begin{kodi}
\obj [square=2.5em] { A & B & C & \phantom{D} \\ };
\mor B -> C;
\mor * -> A;
\end{kodi}
\end{tcblisting}

As a target, {\ttfamily *} evaluates to the tail of the previous chain.

\begin{tcblisting}{snippet, trim}
\begin{kodi}
\obj [square=2.5em] { \phantom{A} & B & C & D \\ };
\mor B -> C;
\mor D -> *;
\end{kodi}
\end{tcblisting}

The natural use case for {\ttfamily *} is chain gluing.

\begin{tcblisting}{snippet, trim}
\begin{kodi}
\obj [square=2.5em] { A & B \\ D & C \\ };
\mor A -> B -> C;
\mor * -> D -> *;
\end{kodi}
\end{tcblisting}

As a source, {\ttfamily +} evaluates to the tail of the previous chain.

\begin{tcblisting}{snippet, trim}
\begin{kodi}
\obj [square=2.5em] { \phantom{A} & B & C & D \\ };
\mor B -> C;
\mor + -> D;
\end{kodi}
\end{tcblisting}

As a target, {\ttfamily +} evaluates to the head of the previous chain.

\begin{tcblisting}{snippet, trim}
\begin{kodi}
\obj [square=2.5em] { A & B & C & \phantom{D} \\ };
\mor B -> C;
\mor A -> +;
\end{kodi}
\end{tcblisting}

The natural use case for {\ttfamily +} is chain extension.

\begin{tcblisting}{snippet, trim}
\begin{kodi}
\obj [square=2.5em] { A & B & C & D \\ };
\mor B -> C;
\mor A -> + -> D;
\end{kodi}
\end{tcblisting}

The meanings of {\ttfamily *} and {\ttfamily +} swap on opposite chains.

Chain extension can be obtained using {\ttfamily *}.

\begin{tcblisting}{snippet, trim}
\begin{kodi}
\obj [square=2.5em] { A & B & C & D \\ };
\mor B <- C;
\mor D -> * -> A;
\end{kodi}
\end{tcblisting}

Chain gluing can be obtained using {\ttfamily +}.

\begin{tcblisting}{snippet, trim}
\begin{kodi}
\obj [square=2.5em] { A & B \\ D & C \\ };
\mor A <- B <- C;
\mor + -> D -> +;
\end{kodi}
\end{tcblisting}

\endgroup

\newpage
\subsection{Names: expansion}
The expansion behaviour of the naming routine can be configured
inside any \CoDi\ scope using the \lstinline!expand! key.

\begin{lstlisting}
/codi/expand = none | once | full
\end{lstlisting}

The three available settings correspond to different degrees of expansion.
A side by side comparison completely illustrates their meanings.

\begin{tcblisting}{snippet, trim}
\begin{codi}
\def\B{Z}
\def\A{\B}
\obj{ |[expand=none]| \A &     % name: A (default)
      |[expand=once]| \A &     % name: B
      |[expand=full]| \A \\ }; % name: Z
\mor A -> B -> Z;
\end{codi}
\end{tcblisting}

\hfill$\therefore$\hfill\null

The default behaviour is to avoid expansion in compliance with the principle
that \emph{names should be predictable from the \emph{literal} code}.
Furthermore, it is seldom wise to liberally expand tokens.

There are circumstances in which it is useful to perform token expansion,
though. A useful application is procedural drawing.

\begin{tcblisting}{snippet, trim}
\begin{codi}
\foreach [count=\r] \l in {A,B,C}
  \foreach [count=\c] \n in {n-1,n,n+1}
    \obj [expand=full] at (3em*\c,-2em*\r) {\l_{\n}};
\mor (A_{n}) -> (B_{n+1}) -> (C_{n}) -> (B_{n-1}) -> (A_{n});
\end{codi}
\end{tcblisting}

In some cases finer control is needed. For instance, full expansion
yields unpractical results when parametrizing macros.

\begin{tcblisting}{snippet, trim}
\begin{codi}
\foreach [count=\c] \m in {\lim,\prod}
  \obj [expand=full] at (4em*\c,0) {\m F};
\mor (lim F) -> (DOTSI prodop slimits@ F);
\end{codi}
\end{tcblisting}

This explains why a setting to force a single expansion exists.

\begin{tcblisting}{snippet, trim}
\begin{codi}
\foreach [count=\c] \m in {\lim,\prod}
  \obj [expand=once] at (4em*\c,0) {\m F};
\mor (lim F) -> (prod F);
\end{codi}
\end{tcblisting}

\newpage
\subsection{Names: replacement}
The character replacement behaviour of the naming routine can be configured
inside any \CoDi\ scope using various keys.

\begin{lstlisting}[style=metacode]
/codi/replace character = <character> with <character>
/codi/replace charcode = <charcode> with <character>
/codi/remove characters = <characters>
/codi/remove character = <character>
/codi/remove charcode = <charcode>
\end{lstlisting}

You can set up a replacement for any character, using the character code for
the hardest to type, like {\ttfamily \textvisiblespace} or {\ttfamily \textbackslash}.

\begin{tcblisting}{snippet, trim}
\begin{codi}[tetragonal]
\obj{ |[replace character=F with G]| \lim F &     % name: lim G
      |[remove character=F]|         \lim F \\    % name: lim
      |[replace charcode=92 with /]| \lim F &     % name: /lim F
      |[remove charcode=32]|         \lim F \\ }; % name: limF
\mor (lim G) -> (lim) -> (/lim F) -> (limF);
\end{codi}
\end{tcblisting}

\hfill$\therefore$\hfill\null

The default behaviour is removal of the minimal set of universally annoying%
\footnote{The difficult part is not creating the names but having to type them.}
characters: {\ttfamily (),.:}  have special meanings to \TikZ\ while
{\ttfamily \textbackslash} is impossible to type by ordinary means, so they're \emph{kaput}.

Each one can be restored by replacing it with itself. Don't.

Another egregiously bad idea is replacing characters with spaces.
It's tempting because it solves a somewhat common edge case.

\begin{tcblisting}{snippet, trim}
\begin{codi}[tetragonal]
\obj{ \beta & F & b\eta \\ };
\mor F -> beta;
\end{codi}
\end{tcblisting}

Since characters in names are literal, this causes whitespace
duplication and names become inaccessible by ordinary means.

\begin{tcblisting}{snippet, trim}
\begin{codi}[tetragonal]
\obj [replace charcode=92 with \space]
  { \beta & b\eta & \beta \eta \\ };
\mor beta -> (b eta) -> (beta \space eta);
\end{codi}
\end{tcblisting}

The wise solution is writing better code.

\begin{tcblisting}{snippet, trim}
\begin{codi}[tetragonal]
\obj{ \beta & F & b \eta \\ };
\mor F -> beta;
\end{codi}
\end{tcblisting}


\newpage
\subsection{Names: overwriting}
The name overwriting behaviour of the naming routine can be configured
inside any \CoDi\ scope using the \lstinline!overwrite! key.

\begin{lstlisting}
/codi/overwrite = false | alias | true
\end{lstlisting}

The three available settings correspond to different naming priorities.
A side by side comparison completely illustrates their meanings.

\begin{tcblisting}{snippet, trim}
\begin{codi}
\obj{ |[overwrite=false] (A')| A &     % names: A'    (default)
      |[overwrite=alias] (B')| B &     % names: B', B
      |[overwrite=true]  (C')| C \\ }; % names:     C
\mor A' -> B';
\mor B  -> C;
\end{codi}
\end{tcblisting}

\hfill$\therefore$\hfill\null

The default behaviour avoids overwriting explicit labels in order
to give you a simple means of naming conflict resolution.

\begin{tcblisting}{snippet, trim}
\begin{codi}[tetragonal]
\obj {        A & |(A')| A \\
       |(Z')| Z &        Z \\ };
\mor A -> A';
\mor Z -> Z';
\end{codi}
\end{tcblisting}

Sometimes you might want an object to have both a literal and a
semantic alias.

\begin{tcblisting}{snippet, trim}
\begin{codi}
\obj [overwrite=alias] { A & |(center)| B & |(right)| C \\ };
\mor A -> B;
\mor center -> right;
\end{codi}
\end{tcblisting}

The hard overwriting behaviour ignores any label except generated
ones; it exists for completeness and debugging purposes.

\newpage
\section{Styles: scopes}
\CoDi\ structures diagrams into five layers implemented with \TikZ.

\begin{center}
\begin{tabular}{ccc}
  \toprule
  \CoDi's & represents an & using \TikZ's \\
  \midrule
  \lstinline|diagram| & (commutative) diagram   & \lstinline|tikzpicture| \\
  \lstinline|layout|  & arrangement of vertices & \lstinline|matrix| \\
  \lstinline|object|  & vertex                  & \lstinline|node| \\
  \lstinline|arrow|   & edge between vertices   & \lstinline|edge| \\
  \lstinline|label|   & label of an edge        & \lstinline|node| \\
  \bottomrule
\end{tabular}
\end{center}

Each layer can be styled using \TikZ\ keys.

Each layer possesses a default style:
\begin{lstlisting}[style=metacode]
/codi/every diagram
/codi/every layout
/codi/every object
/codi/every arrow
/codi/every label
\end{lstlisting}

You can customize them using \TikZ\ key handlers, \eg
\begin{lstlisting}
/codi/every label/.append style={red}
\end{lstlisting}

Each layer possesses a library of commonplace styles:
\begin{lstlisting}[style=metacode]
/codi/diagrams/
/codi/layouts/
/codi/objects/
/codi/arrows/
/codi/labels/
\end{lstlisting}

They are the proper place to find styles and define you own:
\begin{lstlisting}
/codi/arrows/fat/.style={ultra thick}
\end{lstlisting}

Fully scoping keys is usually unnecessary, as \CoDi\ searches for keys
in the library of the layer it's in before falling back to \TikZ\ default
search algorithm. Here's some meta code demonstrating this:

\begin{lstlisting}
\begin{codi}[<diagram keylist>]
  \obj [<layout keylist>] { |[<object keylist>]| a & b \\ };
  \obj [<object keylist>] {x};
  \mor [<label keylist>]:[<arrow keylist>]
     a [<label keylist>]:[<arrow keylist>] b;
\end{codi}
\end{lstlisting}
  

\newpage
\section{Styles: diagrams}
\tikzset{
  versor/.style={ultra thick, black, ->, >=latex, line cap=round},
  checkers/.style={fill=teal!15},
  measure/.style={decorate, decoration={brace, amplitude=4pt, aspect=#1}},
  measure/.default=0.5,
  label/.style={font=\scriptsize, outer sep=.5em, inner sep=0sp},
  dots/.style={fill=teal!50},
  label style/.style={label, outer sep=.2em, font={\ttfamily\tiny}},
  anchor style/.style={inner sep=0sp, outer sep=0sp, anchor=center},
  pin style/.style={pin distance=.75em, pin edge={thin, black, shorten <=3pt}, label style}
}

\begin{marginfigure}[0cm]
  \begin{subfigure}{\marginparwidth}
    \begin{tikzpicture}[codi, tetragonal=base {0.2\marginparwidth} height {0.2*1.618\marginparwidth}]
      % drawing area
      \clip (-2.5,-2.5) rectangle (2.5,2.5);
      % checkers and dots
      \foreach \i in {-3, ..., 2} \foreach \j in {-3, ..., 2} {
        \pgfmathparse{int(1+abs(\i)+abs(\j))}
        \ifodd\pgfmathresult \fill [checkers] (\i,\j) rectangle +(1,1);\fi
        \pgfmathparse{ \i == sign(\i) && \j == sign(\j) && abs(\i) != -abs(\j) }
        \fill [fill/.expanded={\ifnum\pgfmathresult=1 red\else teal!50\fi}]
          (\i,\j) circle (2pt);
      }
      % base circle  
      \fill [red, opacity=0.1] (1,1) -- (-1,1) -- (-1,-1) -- (1,-1) -- cycle;
      % width marker
      \draw
        [measure, decoration=mirror, yshift=-3pt]
        (0,0) -- (1,0)
        node [label style, midway, below, yshift=-3pt] {base};
      % height marker
      \draw
        [measure=0.7, xshift=-3pt]
        (0,0) -- (0,1)
        node [label style, pos=0.7, left, xshift=-3pt] {height};
      % origin
      \coordinate (O) at (0,0);
      \node at (O) [anchor style, pin={[pin style]-135:{O}}] {};
      % neighbours
      \node [anchor style, pin={[pin style]-45:{above=of O}}, above=of O] {};
      \node [anchor style, pin={[pin style]60:{above left=of O}}, above left=of O] {};
      \node [anchor style, pin={[pin style]90:{left=1 of O}}, left=1 of O] {};
      \node [anchor style, pin={[pin style]120:{below right=2 of O}}, below right=2 of O] {};
      \node [anchor style, pin={[pin style]-45:{below left=1 and 2 of O}}, below left=1 and 2 of O] {};
      % versors
      \draw [versor] (0,0) -- (1,0);
      \draw [versor] (0,0) -- (0,1);
    \end{tikzpicture}
    \subcaption*{Tetragonal}\bigskip
  \end{subfigure}
  \begin{subfigure}{\linewidth}
    \begin{tikzpicture}[codi, hexagonal=horizontal side {0.4*\marginparwidth} angle 60]
      % drawing area
      \clip (-2.5,-2.5) rectangle (2.5,2.5);
      % checkers and dots
      \foreach \i in {-4, ..., 2} \foreach \j in {-3, ..., 2} {
        \pgfmathparse{int(abs(\i)+abs(\j))}
        \ifodd\pgfmathresult\else \fill [checkers] (\i,\j) -- +(1,1) -- +(2,0);\fi
        \pgfmathsetmacro\dotsize{isodd(\pgfmathresult)?1:2}
        \fill [fill/.expanded={\ifnum\pgfmathresult=2 red\else teal!50\fi}]
          (\i,\j) circle (\dotsize pt);
      }
      % base circle  
      \fill [red, opacity=0.1]
        (2,0) -- (0,2) -- (-2,0) -- (0,-2) -- cycle;
      % side marker
      \draw [measure, decoration=mirror, yshift=-3pt]
        (0,0) -- (2,0)
        node [label style, midway, below, yshift=-3pt] {side};
      % angle marker
      \draw (0.66*1.7cm, 0)
        arc (0:30:0.66*1.7cm)
        node [label style, above right] {angle}
        arc (30:60:0.66*1.7cm);
      % origin
      \coordinate (O) at (0,0);
      \node at (O) [anchor style, pin={[pin style]-135:{O}}] {};
      % neighbours
      \node [anchor style, pin={[pin style]-45:{above=of O}}, above=of O] {};
      \node [anchor style, pin={[pin style]60:{above left=of O}}, above left=of O] {};
      \node [anchor style, pin={[pin style]90:{left=1 of O}}, left=1 of O] {};
      \node [anchor style, pin={[pin style]120:{below right=2 of O}}, below right=2 of O] {};
      \node [anchor style, pin={[pin style]-45:{below left=1 and 2 of O}}, below left=1 and 2 of O] {};
      % versors
      \draw [versor] (0,0) -- (1,0);
      \draw [versor] (0,0) -- (0,1);
    \end{tikzpicture}
    \subcaption*{Hexagonal (horizontal)}\bigskip
  \end{subfigure}
  \begin{subfigure}{\linewidth}
    \begin{tikzpicture}[codi, hexagonal=vertical side {0.2*\marginparwidth*2/sqrt(3)} angle 60]
      % drawing area
      \clip (-2.5,-2.5) rectangle (2.5,2.5);
      % checkers and dots
      \foreach \i in {-3, ..., 2} \foreach \j in {-4, ..., 3} {
        \pgfmathparse{int(1+abs(\i)+abs(\j))}
        \ifodd\pgfmathresult \fill [checkers] (\i,\j) -- +(1,1) -- +(0,2);\fi
        \pgfmathsetmacro\dotsize{isodd(\pgfmathresult)?2:1}
        \fill [fill/.expanded={\ifnum\pgfmathresult=3 red\else teal!50\fi}]
          (\i,\j) circle (\dotsize pt);
      }
      % base circle  
      \fill [red, opacity=0.1] (2,0) -- (0,2) -- (-2,0) -- (0,-2) -- cycle;
      % side marker
      \draw [measure=0.3, xshift=-3pt]
        (0,0) -- (0,2) node [label style, pos=0.3, left, xshift=-3pt]  {side};
      % angle marker
      \draw 
        ([shift=(30:0.66*0.982cm)]0,0)
        arc (30:60:0.66*0.982cm)
        node [label style, above right] {angle}
        arc (60:90:0.66*0.982cm);
      % origin
      \coordinate (O) at (0,0);
      % neighbours
      % versors
      \draw [versor] (0,0) -- (1,0);
      \draw [versor] (0,0) -- (0,1);
    \end{tikzpicture}
    \subcaption*{Hexagonal (vertical)}\bigskip
  \end{subfigure}
  %% \caption*{Diagram grid examples}
\end{marginfigure}

Diagrams can be laid over regular grids:

\begin{lstlisting}
/codi/diagrams/tetragonal=base <length> height <length>
  (@\hfill@) (default: base 4.5em height 2.8em)
\end{lstlisting}

\begin{lstlisting}
/codi/diagrams/hexagonal=<direction> side <length> angle <angle>
  (@\hfill@) (default: horizontal side 4.5em angle 60)
\end{lstlisting}

When one of these keys is used
\begin{itemize}[noitemsep]
  \item the versors of the \NiceURL{coordinate system}{http://texdoc.net/texmf-dist/doc/generic/pgf/pgfmanual.pdf\#page=357} are changed,
  \item the \NiceURL{node positioning}{http://texdoc.net/texmf-dist/doc/generic/pgf/pgfmanual.pdf\#page=229} is set up to lay them on grid,
  \item and the corresponding key will be applied to all layouts.
\end{itemize}

The pictures show the key parameters, versors, and a unitary grid.

This setup allows you to mix coordinates and
relative positioning keys to arrange objects.

As usual, relative positioning keys can accept two components, a radius,
or nothing at all (which defaults to a certain radius).

When using a radius (or defaulting to $1$)
the tetragonal grid uses Manhattan distance to
lay objects along concentric rectangles.

When using a radius (or defaulting to $2$)
the hexagonal grid%
\footnote{which in truth is built upon a tetragonal grid}
uses Chebyshev distance to
lay objects along concentric rhombi.

To clarify, a few relative positioning keys are drawn along
with red zones displaying the default radii around the origins.

\newpage
\section{Styles: layouts}
Layouts can be laid over regular grids:

\begin{lstlisting}
/codi/layouts/tetragonal=base <length> height <length>
  (@\hfill@) (default: base 4.5em height 2.8em)
\end{lstlisting}

\begin{lstlisting}
/codi/layouts/hexagonal=<direction> side <length> angle <angle>
  (@\hfill@) (default: horizontal side 4.5em angle 60)
\end{lstlisting}

When one of these keys is used the layout columns and rows will be
spaced and offset in order to reproduce the grids given by diagram styles.

\begin{tcblisting}{snippet, trim={2 and -1}}
\begin{codi}
  \obj [hexagonal=horizontal side 1.5em angle 60] {
    A & B &   \\
    C & D & E \\
    F & G &   \\
  };
\end{codi}
\end{tcblisting}

\begin{tcblisting}{snippet, trim={2 and -1}}
\begin{codi}
  \obj [hexagonal=vertical side 1.5em angle 60] {
    A & C & F \\
    B & D & G \\
      & E &   \\
  };
\end{codi}
\end{tcblisting}

Note that \emph{each row must have the same number of cells}%
\footnote{this is different from the behaviour of, say, tables}
or the spacing will be incorrect.

\newpage
\section{Styles: objects}
No styles are available at the moment.

\newpage
\section{Styles: arrows}
\begin{lstlisting}
/codi/arrows/crossing over
/codi/arrows/crossing over/clearance=<length> (@\hfill@) (default: 0.5ex)
/codi/arrows/crossing over/color=<color> (@\hfill@) (default: white)
\end{lstlisting}

This key a provides the configurable illusion of an arrow passing
over a \emph{previously drawn} one.

\begin{tcblisting}{snippet, trim={3 and -1}}
\begin{codi}[tetragonal=base 4.5em height 1em]
\obj { A & B \\ D & C \\};
\mor A -> C;
\mor :[crossing over] D -> B;
\end{codi}
\end{tcblisting}

%%%%%%%%%%%%%%%%%%%%%%%%%%%%%%%%%%%%%%%%%%%%%%%%%%%%%%%%%%%%%%%%%%%%%%%%%%%%%%%%

\begin{lstlisting}
/codi/arrows/slide=<length>
\end{lstlisting}

This key slides an arrow backward (negative) and forward (positive) along its direction of the given length.

\begin{tcblisting}{snippet, trim={3 and -1}}
\begin{codi}[tetragonal=base 4.5em height 1em]
\obj { A & B \\ C & D \\ E & F \\ };
\mor :[slide=-.3em, red] A -> B;
\mor C -> D;
\mor :[slide=+.3em, blue] E -> F;
\end{codi}
\end{tcblisting}

%%%%%%%%%%%%%%%%%%%%%%%%%%%%%%%%%%%%%%%%%%%%%%%%%%%%%%%%%%%%%%%%%%%%%%%%%%%%%%%%

\begin{lstlisting}
/codi/arrows/shove=<length>
\end{lstlisting}

This key shoves an arrow to the left (negative) and to the right (positive) with respect to its direction of the given length.

\begin{tcblisting}{snippet, trim={3 and -1}}
\begin{codi}
\obj { A & B \\ };
\mor :[shove=-.3em, red] A -> B;
\mor A -> B;
\mor :[shove=+.3em, blue] A -> B;
\end{codi}
\end{tcblisting}

\newpage
\section{Styles: labels}
\begin{lstlisting}[style=metacode]
/codi/labels/mid
\end{lstlisting}

This key places a label in the middle of an arrow.

\begin{tcblisting}{snippet, trim={3 and -1}}
\begin{codi}
\obj { A & B \\ };
\mor [mid] A f:-> B;
\end{codi}
\end{tcblisting}

\newpage
\section{Gallery}
% \makeatletter
% \newcommand\prev{\the\tikz@lastxsaved,\the\tikz@lastysaved}
% \makeatother

The remainder of the text is just commented examples.

\clearpage

\subsection{2-cells}

\begin{tcblisting}{gallery}
% From the LaTeX preamble:
%   \usepackage{tikz-cd}

% We use tikz-cd keys to quickly whip up an arrow style for a 2-morphism.
\pgfqkeys{/codi/arrows}{
  -2>/.style={
      /tikz/commutative diagrams/Rightarrow,
      /tikz/commutative diagrams/shorten=2pt,
  }
}

\begin{codi}
  \obj{ X & Y & Z \\ };
  \mor[swap]:[bend right] X f:-> Y m:-> Z;
  \mor      :[bend left]  X g:-> Y n:-> Z;
  \mor       f \alpha:-2> g;
  \mor[swap] m  \beta:-2> n;
\end{codi}
\end{tcblisting}

\subsection{Snake}

\begin{tcblisting}{gallery}
\begin{codi}[tetragonal]
  \obj{          & \ker a   & \ker b   & \ker c   &   \\
                 & A        & B        & C        & 0 \\
        |(0')| 0 & A'       & B'       & C'       &   \\
                 & \coker a & \coker b & \coker c &   \\ };

  \mor   (ker a) ->   (ker b) ->   (ker c);
  \mor (coker a) -> (coker b) -> (coker c);
  \mor       A  f :-> B  g :-> C -> 0;
  \mor 0' -> A' f':-> B' g':-> C';

  \mor[near start] (ker a) -> A a:-> A' -> (coker a);
  \mor[near start] (ker b) -> B b:-> B' -> (coker b);
  \mor[near start] (ker c) -> C c:-> C' -> (coker c);

  \draw[/codi/arrows/crossing over, ->, rounded corners, >=stealth]
    (ker c) -- ++( 0.6,0) -- ++(0,-1.55)
            -- ++(-3.2,0) -- ++(0,-1.45) -- (coker a);
\end{codi}
\end{tcblisting}

\clearpage

\subsection{The fourth associahedron}

\begin{tcblisting}{gallery,listing options={literate={×}{{\texttimes}}1 {ı}{{\i}}1}}
\begin{codi}
% From the LaTeX preamble:
%   \usepackage{newunicodechar}
%   \newunicodechar{ı}{\mathbf 1}
%   \newunicodechar{×}{\otimes}

  \foreach [count=\n] \o in {
      ((w×x)×y)×x,
      (w×(x×y))×x,
      w×((x×y)×x),
      w×(x×(y×x)),
      (w×x)×(y×x)
    } \obj (\n) at (72*\n:7em) {\o};

  \mor 1 "a_{w,x,y}×ı_z": -> 2
           "a_{w,x×y,z}": -> 3
         "ı_w×a_{x,y,z}": -> 4;
  \mor *   "a_{w×x,y,z}": -> 5
           "a_{w,x,y×z}": -> *;
\end{codi}
\end{tcblisting}

\clearpage

\subsection{Pullback \& pushout}

\begin{tcblisting}{gallery}
\begin{codi}[hexagonal]
  \obj{ |(pb)| A \times_Z B & B \\
               A            & Z \\ };
  \obj[above left=of pb] {Q};

  \mor[swap] pb p_1:-> A f:-> Z;
  \mor        * p_2:-> B g:-> *;

  \mor[swap]:[bend right] Q q_1:-> A;
  \mor      :[bend left]  * q_2:-> B;
  \mor [mid]:[dashed]     *   u:-> pb;
\end{codi}
\end{tcblisting}

\begin{tcblisting}{gallery}
\begin{codi}[hexagonal]
  \obj{ Z &                   B \\
        A & |(po)| A \sqcup_Z B \\ };
  \obj[below right=of po] {Q};

  \mor[swap] Z f:-> A i_1:-> po;
  \mor       * g:-> B i_2:-> *;

  \mor[swap]:[bend right]  A j_1:-> Q;
  \mor      :[bend left]   B j_2:-> *;
  \mor [mid]:[dashed]     po   u:-> *;
\end{codi}
\end{tcblisting}

\clearpage

\subsection{Complexes sequence}

\begin{tcblisting}{gallery}
\begin{codi}
  \obj (M) {   & \vdots  & \vdots  & \vdots  &   \\
             0 & A_{n+1} & B_{n+1} & C_{n+1} & 0 \\
             0 & A_{n}   & B_{n}   & C_{n}   & 0 \\
             0 & A_{n-1} & B_{n-1} & C_{n-1} & 0 \\
               & \vdots  & \vdots  & \vdots  &   \\ };

  \foreach \n/\row in {n+1/2, n/3, n-1/4}
    \mor (M-\row-1) -> (A_{\n}) "\alpha_{\n}":-> (B_{\n})
                                 "\beta_{\n}":-> (C_{\n}) -> (M-\row-5);

  \foreach \l/\col/\q in {A/2/, B/3/', C/4/''}
    \mor (M-1-\col) -> (\l_{n+1}) "\partial\q_{n+1}":-> (\l_{n})
                                  "\partial\q_{n}"  :-> (\l_{n-1}) -> (M-5-\col);
\end{codi}
\end{tcblisting}

\clearpage

\subsection{Braid}

% TODO: there's a bug here. If one does \begin{codi}[ l/.style={bend left}
% TODO: (note the space before l) then it's not recognized. Investigate.
\begin{tcblisting}{gallery}
\begin{codi}[l/.style={bend left}, r/.style={bend right} ]
  \obj [ hexagonal=horizontal side 6em angle 45, remove characters=H_\{q+\} ] {
    H_{q+2}(X)   & H_{q+2}(X,Y) & H_{q+1}(Y,Z) & H_{q}(Z)     \\
    H_{q+2}(Y)   & H_{q+2}(X,Z) & H_{q+1}(Y)   & H_{q+1}(X,Z) \\
    H_{q+2}(Y,Z) & H_{q+1}(Z)   & H_{q+1}(X)   &              \\
  };
  
  \mor :[blue]  2Y  -> 2X  l,-> 2XY   -> 1Y  -> 1X;
  \mor :[green] 2Y  -> 2YZ r,-> 1Z    -> 1Y  -> 1YZ l,-> Z;
  \mor :[cyan]  2X  -> 2XZ   -> 1Z  r,-> 1X  -> 1XZ   -> Z;
  \mor :[red]   2YZ -> 2XZ   -> 2XY l,-> 1YZ -> 1XZ;
\end{codi}
\end{tcblisting}

\clearpage

\subsection{Hammock}

\begingroup\catcode`~=12
\begin{tcblisting}{gallery}
\begin{codi}[x=4em, y=-3em, node distance=1 and 1,
    sim/.style={sloped, auto,
      edge node={node[every edge quotes][/velos/install quote
        handler,"\sim", anchor=south, outer sep=-.15em]}
    },
    ~>/.style={->, sim},
    <~/.style={<-, sim},
    ../.style={line width=.25ex, dash pattern=on 0sp off .75ex, line cap=round},
    remove characters=_\{\},
    expand=full,
  ]
    
  \foreach [count=\c] \col in {1, 2, 3, n}
  \foreach [count=\r] \row in {K_{\col}, C_{0\col}, \vdots, C_{m\col}, L_{\col}}
    \obj [name/.expanded={\ifnum\r=3 vdots\col\fi}] at (\c,\r) {\row};

  \obj  [left=of vdots1] {X};
  \obj [right=of vdotsn] {Y};

  \foreach \col in {1, 2, 3, n}
    \mor (K\col) ~> (C0\col) ~> (vdots\col) ~>  (Cm\col) ~> (L\col);
  
  \foreach \row in {K, C0, Cm, L} {
    \mor (\row1) -> (\row2) <~ (\row3) .. (\row n);
    \mor X <~ + -> Y;
  }
\end{codi}
\end{tcblisting}
\endgroup



% \clearpage

\null\vfill

\tikzset{
  versor/.style={ultra thick, black, ->, >=latex, line cap=round},
  checkers/.style={fill=teal!15},
  measure/.style={decorate, decoration={brace, amplitude=4pt}},
  label/.style={font=\scriptsize, outer sep=.5em, inner sep=0sp},
  dots/.style={fill=teal!50},
}

\marginpar{\begin{SmashAndCenter}\begin{kodi}[comb=horizontal step 3.2em angle 60]
  \clip (-1.5,-1.5) rectangle (3.5,2.5);
      
  \foreach \i in {-3, ..., 3} \foreach \j in {-2, ..., 2} {
    \pgfmathparse{int(1+abs(\i)+abs(\j))}
    \ifodd\pgfmathresult \fill [checkers] (\i,\j) -- +(1,1) -- +(2,0);\fi
    \pgfmathsetmacro\dotsize{isodd(\pgfmathresult)?2:1}
    \fill [dots] (\i,\j) circle (\dotsize pt);
  }
  
  \draw [measure, decoration=mirror, yshift=-4pt]
    (0,0) -- (2,0) node [label, midway, below] {step};

  \draw (.75,.75) arc (60:30:2.4em) coordinate (wot) arc (30:0:2.4em);
  \path [late options={name=wot,pin={[pin edge={thin, black}, label, right,
    outer sep=.2em]30:angle}}];

  \draw [versor] (0,0) -- (1,0);
  \draw [versor] (0,0) -- (0,1);
\end{kodi}\end{SmashAndCenter}}

\vfill

\marginpar{\begin{SmashAndCenter}\begin{kodi}[comb=vertical step 3.2em angle 60]
  \clip (-1.5,-1.5) rectangle (1.5,3.5);

  \foreach \i in {-2, ..., 2} \foreach \j in {-3, ..., 3} {
    \pgfmathparse{int(1+abs(\i)+abs(\j))}
    \ifodd\pgfmathresult \fill [checkers] (\i,\j) -- +(1,1) -- +(0,2);\fi
    \pgfmathsetmacro\dotsize{isodd(\pgfmathresult)?2:1}
    \fill [dots] (\i,\j) circle (\dotsize pt);
  }

  \draw [measure, xshift=-4pt]
    (0,0) -- (0,2) node [label, midway, left]  {step};

  \draw (.75,.75) arc (30:60:2.4em) coordinate (wot) arc (60:90:2.4em);
  \path [late options={name=wot,pin={[pin edge={thin, black}, label, right,
    outer sep=.2em]30:angle}}];

  \draw [versor] (0,0) -- (1,0);
  \draw [versor] (0,0) -- (0,1);
\end{kodi}\end{SmashAndCenter}}

\vfill

\marginpar{\begin{SmashAndCenter}\begin{kodi}[rectangular={2em}{2.5em}]
  \clip (-1.5,-1.5) rectangle (2.5,2.5);
      
  \foreach \i in {-2, ..., 2} \foreach \j in {-2, ..., 2} {
    \pgfmathparse{int(1+abs(\i)+abs(\j))}
    \ifodd\pgfmathresult \fill [checkers] (\i,\j) rectangle +(1,1);\fi
    \fill [dots] (\i,\j) circle (2pt);
  }
  
  \draw [measure, decoration=mirror, yshift=-4pt]
    (0,0) -- (1,0) node [label, midway, below] {base};
  
  \draw [measure, xshift=-4pt]
    (0,0) -- (0,1) node [label, midway, left] {height};
  
  \draw [versor] (0,0) -- (1,0);
  \draw [versor] (0,0) -- (0,1);
\end{kodi}\end{SmashAndCenter}}



\vfill

\begin{kodi}[grid={3em}{3em}, metric={2}{0.5}]
  \foreach \i in {-1, ..., 5}
  \foreach \j in {-1, ..., 5}
    \fill (\i,\j) circle (1pt);  
  
  \draw [versor] (0,0) -- (1,0);
  \draw [versor] (0,0) -- (1,1);
  \draw [versor] (0,0) -- (0,1);

  \node (O) {O};
  \node [above      =  of O] {X};
  \node [above right=  of O] {X};
  \node [right      =  of O] {X};
  \node [above      =1 of O] {1};
  \node [above right=1 of O] {1};
  \node [right      =1 of O] {1};
  \node [above      =2 of O] {2};
  \node [above right=2 of O] {2};
  \node [right      =2 of O] {2};
  \node [above      =3 of O] {3};
  \node [above right=3 of O] {3};
  \node [right      =3 of O] {3};
  \node [above      =4 of O] {4};
  \node [above right=4 of O] {4};
  \node [right      =4 of O] {4};
\end{kodi}


\end{document}
