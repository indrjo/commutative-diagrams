%==[ document class ]===========================================================

\documentclass[12pt]{scrbook}

%==[ languages ]================================================================

\usepackage[british]{babel}
% \hyphenation{Fortran hy-phen-ation}

%==[ fonts, encoding ]==========================================================

\usepackage{libertine}
\usepackage{libertinust1math}
\usepackage[ttdefault=true]{AnonymousPro}

\usepackage[utf8]{inputenc}
\usepackage[T1]{fontenc}

%==[ page geometry ]============================================================

\usepackage{geometry}

\geometry{
  a4paper,
  portrait,
  marginparwidth=4.25cm,
  marginparsep=.75cm,
  %% showframe,
  width=11cm,
  hmarginratio=10:25,
  height=23cm,
  vmarginratio=20:30,
}

\makeatletter
\let\org@Gm@pageframes\Gm@pageframes
\renewcommand*{\Gm@pageframes}{%
  \begingroup
    \color{gray}%
    \org@Gm@pageframes
  \endgroup
}
\makeatother

%==[ page layout ]==============================================================

%\usepackage{multicol}
%\usepackage{tabularx}
\usepackage{booktabs}
%\usepackage{verbatimbox}
%\usepackage{wrapfig}

\usepackage{float}
% \usepackage{framed}
% \usepackage{lipsum}
\usepackage{enumitem}
% \usepackage{pdfpages}

% \usepackage{caption}
\usepackage{sidenotes}

% \DeclareCaptionStyle{sidenote}{font=scriptsize}

% \usepackage{todonotes}
% \usepackage{marginnote}

\usepackage[
  headwidth=textwithmarginpar,
  footwidth=textwithmarginpar,
]{scrlayer-scrpage}
%\clearpairofpagestyles

% \lofoot{}
% \cofoot{}
% \rofoot{}
% \lohead{}
% \cohead{}
% \rohead{\thepage}

%==[ page style ]===============================================================

\setkomafont{disposition}{\rmfamily\scshape}

\usepackage{parskip}

% \usepackage{csquotes}

\usepackage{caption}
\usepackage{subcaption}

\DeclareCaptionStyle{marginfigure}{
  font=footnotesize,
  justification=centering,
}

\captionsetup[subfigure]{
  textfont=footnotesize,
  %% singlelinecheck=off,
  justification=centering
}

\setcounter{secnumdepth}{0}

%==[ commonplace math ]=========================================================

\usepackage{mathtools}
\usepackage{amsfonts}
\usepackage{amssymb}

\DeclareMathOperator{\coker}{coker}

% \def\id{{\mathbf 1}}
% \DeclareUnicodeCharacter{0131}{{\noexpand\mathbf R}}
% \DeclareUnicodeCharacter{00D7}{{\otimes}}
% \catcode`ı=\active \letı\id
% \catcode`×=\active \let×\otimes
% NOTE: needed just for the associahedron example
\usepackage{newunicodechar}
\newunicodechar{ı}{\mathbf 1}
\newunicodechar{×}{\otimes}

%==[ graphics ]=================================================================

\usepackage{graphicx}
\usepackage{xcolor}
% TODO: what did I need soul for again?
\usepackage{color,soul}

\usepackage{tikz}
\usetikzlibrary{decorations.pathreplacing}

%==[ koDi ]=====================================================================

% \usetikzlibrary{kodi}
\usepackage{kodi}
\usetikzlibrary{positioning}

%==[ alternatives to koDi ]=====================================================

% So, yeah, I got *really* tired...
\usepackage{filecontents}

% Copied from Xy sourcecode
\def\Xy{\hbox{\kern-.1em X\kern-.3em\lower.4ex\hbox{Y\kern-.15em}}}

%==[ utilities and custom macros ]==============================================

\usepackage{etoolbox}
\usepackage{xparse}

% \NewDocumentCommand{\SmashAndCenter}{m}{%
% \begingroup\setbox0=\hbox{#1}%
% \newdimen\tmp\tmp=\dimexpr-.5\ht0+.5\dp0\relax%
% \hfill\smash{\raisebox{\the\tmp}{\box0}}\hfill\null\endgroup%
% }

\newsavebox\SmashAndCenterBox
\newdimen\SmashandCenterRaise

% TODO: make overflow on left pages go left
\NewDocumentEnvironment{SmashAndCenter}{}{%
\begin{lrbox}{\SmashAndCenterBox}}{%
\end{lrbox}%
\SmashandCenterRaise=\dimexpr-.5\ht\SmashAndCenterBox+.5\dp\SmashAndCenterBox\relax%
\hfill\smash{\raisebox{\the\SmashandCenterRaise}{\usebox\SmashAndCenterBox}}\hfill\null%
}

\def\eg{e.\,g.}
%% \deg\ie{i.\,e.}

%==[ sourcecode and examples ]==================================================

\usepackage{listings}

% http://tex.stackexchange.com/a/336331/82186
\makeatletter
\lst@Key{lastline}\relax{\ifnumcomp{#1}{<}{0}{%
  \let\mylst@file\lst@intname\sbox0{\lstinputlisting{\mylst@file}}%
  \def\lst@lastline{\the\numexpr#1+\value{lstnumber}-1\relax}}%
  {\def\lst@lastline{#1\relax}}}
\makeatother

\usepackage{showexpl}

\makeatletter
\lst@Key{postset}\relax{\def\SX@postset{#1}}
\newcommand\SX@postset{}
\renewcommand*\SX@resultInput{%
  \ifx\SX@graphicname\@empty
    \begingroup
      \MakePercentComment\makeatother\catcode`\^^M=5\relax
      \SX@@preset\SX@preset
      \if@SX@rangeaccept
       \let\SX@tempa=\SX@input
      \else
       \let\SX@tempa=\input
      \fi
      \if\SX@scaled ?%
        \let\SX@tempb=\@firstofone
      \else
        \if\SX@scaled !%
          \def\SX@tempb##1{\resizebox{\SX@width}{!}{##1}}%
        \else
          \def\SX@tempb##1{\scalebox{\SX@scaled}{##1}}%
        \fi
      \fi
      \SX@tempb{\SX@tempa{\SX@codefile}}\SX@postset\par
    \endgroup
  \else
    \expandafter\includegraphics\expandafter[\SX@graphicparam]%
      {\SX@graphicname}%
  \fi
}
\makeatother

% \lstdefinelanguage{TikZ}{
%   morekeywords={for},
%   sensitive=false,
%   morecomment=[l]{//},
%   morecomment=[s]{/*}{*/},
%   morestring=[b]",
% }

\lstdefinestyle{TeX}{
  language=[LaTeX]TeX,
  basicstyle=\ttfamily\lst@ifdisplaystyle\scriptsize\fi,
  backgroundcolor=\color{teal!10},
  % keywordstyle=*\color{blue},
  % identifierstyle=\color{orange}\bfseries,
  % morekeywords={\path},
  alsoother={@},
  moretexcs={
    \starttext,\stoptext,\usemodule,
    \tikzpicture,\endtikzpicture,
    \tikzexternalize,
    \starttikzpicture,\stoptikzpicture,
    \usetikzlibrary,
    \kodi,\endkodi,
    \startkodi,\stopkodi,
    \lay,\obj,\mor,
    \bye,
    \draw,\foreach,
    \ar,
    \psset,\everypsbox,\ncline,\ncarc,
    \xymatrix,
  },
  texcsstyle=*\bfseries,
  % morestring=[b]",
  commentstyle=\itshape\color{black!60},
  frame=none,
  % extendedchars=false,
  inputencoding=utf8,
  escapeinside={(@}{@)},
  moredelim=**[is][\color{orange!80!black}]{@opt@}{@/opt@},
  moredelim=**[is][\color{blue!80!black}]{@nws@}{@/nws@},
  % moredelim=**[is][\color\underbar]{@rep@}{@/rep@},
  moredelim=**[s][\itshape]{<}{>},
  % NOTE: × and ı are needed just for the associahedron example
  literate={~}{{\textvisiblespace}}1 {XOR}{{$\vert$}}1 {:}{{\textbf{:}}}1 {:=}{{$\equiv$}}1 {×}{{\texttimes}}1 {ı}{{\i}}1,
}

\lstset{style=TeX}
% \lstMakeShortInline[style=TeX]"

% \makeatletter
% \newcount\ublvl\ublvl=0
% \newcount\ubdpt\ubdpt=0
% \newdimen\ubgap\ubgap=.2em
% \def\underbra#1{\underline {\sbox \tw@ {\global\advance\ubdpt1\advance\ublvl1#1}\dp \tw@ \dimexpr\ubgap*(\ubdpt-\ublvl-1)\relax \box \tw@ }\ifnum\ublvl=0\ubdpt=0\fi}
% \makeatother

% \lstset{explpreset={
%   wide,
%   basicstyle=\ttfamily\scriptsize,
%   pos=o,
%   width=\marginparwidth,
%   hsep=\marginparsep,
%   rframe={},
%   preset={\centering\tikzpicture[kodi]},
%   postset={\endtikzpicture}
% }}

\usepackage{tcolorbox}
\tcbuselibrary{listingsutf8}
\tcbset{listing utf8=latin1}

\tcbset{
  smash and center/.style={
    if odd page={
      before lower=\begin{SmashAndCenter},
      after lower=\end{SmashAndCenter}
    }{
      before upper=\begin{SmashAndCenter},
      after upper=\end{SmashAndCenter}
    }},
  trim/.style args={#1 and #2}{
    listing options={
      firstline=#1,
      lastline=#2}},
  trim/.default={2 and -1},
  snippet/.style={
    size=tight,
    colback=white,
    colframe=white,
    if odd page={
      listing side text,
      lefthand width=\textwidth,
      righthand width=\marginparwidth,
      halign lower=center,
    }{
      text side listing,
      righthand width=\textwidth,
      lefthand width=\marginparwidth,
      halign upper=center,
    },
    toggle enlargement,
    grow to right by=\marginparsep+\marginparwidth,
    sidebyside gap=\marginparsep,
    smash and center,
    listing options={}
  },
  commented snippet/.style={
    snippet,
    comment={}, % avoids text compilation (possible tcb bug?)
    if odd page={
      listing side comment,
    }{
      comment side listing,
    },
  },
  gallery/.style={
    size=tight,
    colback=white,
    colframe=white,
    if odd page={
      % listing side text,
      % lefthand width=\textwidth,
      % righthand width=\marginparwidth,
      grow to right by=\marginparsep+\marginparwidth,
    }{
      grow to right by=\marginparsep+\marginparwidth,
      % text side listing,
      % righthand width=\textwidth,
      % lefthand width=\marginparwidth,
      % halign upper=center,
    },
    toggle enlargement,
    % grow to right by=\marginparsep+\marginparwidth,
    text above listing,
    listing options={}
  }
}

% \tcbset{kodi snippet/.style={snippet, trim, smash and center}}

\def\nilstrut{\rule{0sp}{0sp}}

%==[ extra stuff ]==============================================================

\usepackage{soul}

% \usepackage{lipsum}

\usepackage{hologo}
\def\ConTeXt{\hologo{ConTeXt}}
\def\koDi{{\scshape koDi}}
\def\TikZ{{\scshape TikZ}}

% \usepackage[lastpage,user]{zref}
\usepackage[
  % colorlinks=true,
  % urlcolor=blue,
  % linkbordercolor=black,
  % pdfborderstyle={/S/U/W .4}% border style will be underline of width 1pt
  % urlbordercolor=cyan        % color of external links
  hidelinks
]{hyperref}

\def\NiceURL#1#2{\href{#2}{\color{blue}\ul{#1}}}

%==[ microtype ]================================================================

\usepackage{microtype}


\begin{document}

%==[ TITLE PAGE ]===============================================================

\thispagestyle{empty}
\noindent
\resizebox{\linewidth}{!}{\scshape koDi}\\[0.62em]
\resizebox{\linewidth}{!}{\scshape kommutative Diagramme für \TeX}\\[1.62em]
\resizebox{\linewidth}{!}{\scshape enchiridion}\par
\vfill
\marginpar{
  \resizebox{\linewidth}{!}{\scshape\color{red} unreleased}\\[0.62em]
  \resizebox{\linewidth}{!}{\scshape v1.0.0}\\
  \resizebox{\linewidth}{!}{\scshape \today}
}

%==[ FOREWORD ]=================================================================

\newpage
% \begin{adjustwidth}{.45\textwidth}{.45\textwidth-\marginparwidth-\marginparsep}
\noindent\koDi\ is a \TikZ\ library. Its purpose
is drawing commutative diagrams.
It is designed precisely to overcome
the shortcomings of traditional ones.
The syntax is minimalistic and intelligible.\par
\hfill{\itshape Paolo al-Imkānī Brasolin}
% \end{adjustwidth}

%==[ PRELIMINARIES ]============================================================

\newpage
\section{Preliminaries}

% \TikZ\ is the only requirement of \koDi.  This ensures compatibility with
most \TeX\ flavours.  Here are minimal working examples for the main dialects:

\begin{figure}[H]
  \begin{adjustwidth}{0sp}{-\marginparwidth-\marginparsep}
    \begin{subfigure}{\marginparwidth}
      \caption*{\TeX}
      \begin{lstlisting}

\input kodi

\kodi
  % diagram here
\endkodi
\bye
      \end{lstlisting}
    \end{subfigure}
    \hfill
    \begin{subfigure}{\marginparwidth}
      \caption*{\ConTeXt\ module}
      \begin{lstlisting}

\usemodule[kodi]
\starttext
\startkodi
  % diagram here
\stopkodi
\stoptext
      \end{lstlisting}
    \end{subfigure}
    \hfill
    \begin{subfigure}{\marginparwidth}
      \caption*{\LaTeX\ package}
      \begin{lstlisting}
\documentclass{article}
\usepackage{kodi}
\begin{document}
\begin{kodi}
  % diagram here
\end{kodi}
\end{document}
      \end{lstlisting}
    \end{subfigure}
    \par
    \begin{subfigure}{\marginparwidth}
      \caption*{\TeX\ (\TikZ\ library)}
      \begin{lstlisting}

\input tikz
\usetikzlibrary kodi

\tikzpicture[kodi]
  % diagram here
\endtikzpicture
\bye
      \end{lstlisting}
    \end{subfigure}
    \hfill
    \begin{subfigure}{\marginparwidth}
      \caption*{\ConTeXt\ (\TikZ\ library)}
      \begin{lstlisting}

\usemodule[tikz]
\usetikzlibrary[kodi]
\starttext
\starttikzpicture[kodi]
  % diagram here
\stoptikzpicture
\stoptext
      \end{lstlisting}
    \end{subfigure}
    \hfill
    \begin{subfigure}{\marginparwidth}
      \caption*{\LaTeX\ (\TikZ\ library)}
      \begin{lstlisting}
\documentclass{article}
\usepackage{tikz}
\usetikzlibrary{kodi}
\begin{document}
\begin{tikzpicture}[kodi]
  % diagram here
\end{tikzpicture}
\end{document}
      \end{lstlisting}
    \end{subfigure}
  \end{adjustwidth}
\end{figure}

A useful \TikZ\ feature exclusive to \LaTeX\ is externalization.
A small expedient is necessary to use it with \koDi.
\begin{marginfigure}[-2em]
  \caption*{\TikZ\ externalization}
  \begin{lstlisting}
\documentclass{article}
\usepackage{tikz}
\usetikzlibrary{kodi}
\usetikzlibrary{external}
\tikzexternalize
  [prefix=tikzpicfolder/]
\begin{document}
\begin{tikzpicture}[kodi]
  % diagram here
\end{tikzpicture}
\end{document}
  \end{lstlisting}
\end{marginfigure}


%==[ QUICK TOUR ]===============================================================

\newpage
\section{Quick tour}

% Objects are typeset using the \lstinline|\obj| macro.

\begin{tcblisting}{snippet, trim}
\begin{codi}
\obj {X};
\end{codi}
\end{tcblisting}

Almost every diagram is laid along a regular grid,
so the customary tabular syntax of \TeX\ is recognized.

\begin{tcblisting}{snippet, trim}
\begin{codi}
\obj {
  A & B \\
  C & D \\
};
\end{codi}
\end{tcblisting}

\CoDi\ objects are self-aware and clever enough to name themselves
so you can comfortably refer to them.

\begin{tcblisting}{snippet, trim}
\begin{codi}
\obj {\lim F};
\draw (lim F) circle (4ex);
\end{codi}
\end{tcblisting}

Morphisms are typeset using the \lstinline!\mor! macro.

\begin{tcblisting}{snippet, trim}
\begin{codi}
\obj { A & B \\ };
\mor A f:-> B;
\end{codi}
\end{tcblisting}

Commutative diagrams exist to illustrate composition and commutation,
so \CoDi\ allows arrow chaining and chain gluing.

\begin{tcblisting}{snippet, trim}
\begin{codi}
\obj { A & B \\ C & D \\ };
\mor A -> B -> D;
\mor * -> C -> *;
\end{codi}
\end{tcblisting}

These are the only two macros defined by \CoDi.

There are more features, though.\\
Read on if this caught your attention.


%==[ ALTERNATIVES ]=============================================================

\newpage
\section{Alternatives}

It is only fair to mutely offer a comparison with mainstream packages,
showing idiomatic code to draw the same diagram.

Let \NiceURL
  {\Xy-pic}
  {http://texdoc.net/texmf-dist/doc/generic/xypic/xyrefer.pdf\#page=1}
set the bar with a \emph{verbatim} extract from its manual.

\begin{tcblisting}{snippet}
\xymatrix{
 U \ar@/_/[ddr]_y \ar[dr] \ar@/^/[drr]^x \\
  & X \times_Z Y \ar[d]^q \ar[r]_p
                 & X \ar[d]_f            \\
  & Y \ar[r]^g & Z                       }
\end{tcblisting}

Here is an example adapted from \NiceURL
  {\ttfamily\small pst-node}
  {http://texdoc.net/texmf-dist/doc/generic/pst-node/pst-node-doc.pdf\#page=23}'s
documentation.

\begin{tcblisting}{snippet}
$ \psset{colsep=2.5em, rowsep=2em}
 \begin{psmatrix}
  U \\
 & X\times_Z Y & X \\
 & Y & Z
 \psset{arrows=->, nodesep=3pt}
 \everypsbox{\scriptstyle}
 \ncline{1,1}{2,2}
 \ncarc[arcangle=-10]{1,1}{3,2}_{y}
 \ncarc[arcangle=10]{1,1}{2,3}^{x}
 \ncline{2,2}{3,2}>{q}
 \ncline{2,2}{2,3}_{p}
 \ncline{2,3}{3,3}<{f}
 \ncline{3,2}{3,3}^{g}
 \end{psmatrix}$
\end{tcblisting}

Next one is refitted from the guide to \NiceURL
  {\ttfamily\small\{tikz-cd\}}
  {http://texdoc.net/texmf-dist/doc/latex/tikz-cd/tikz-cd-doc.pdf\#page=3}.

\begin{tcblisting}{snippet}
\begin{tikzcd}[column sep=scriptsize, row sep=scriptsize]
  U
  \arrow[drr, bend left=10, "x"]
  \arrow[ddr, bend right=10, swap, "y"]
  \arrow[dr] & & \\
    & X \times_Z Y \arrow[r, swap, "p"] \arrow[d, "q"]
      & X \arrow[d, swap, "f"] \\
    & Y \arrow[r, "g"]
      & Z
\end{tikzcd}
\end{tcblisting}

Finally, \textbf{\koDi}.

\begin{tcblisting}{snippet}
\begin{kodi}[golden]
  \obj { |(pb)| X \times_Z Y & X \\
                           Y & Z \\ };
  \obj [above left=of pb] {U};

  \mor[swap] pb p:-> X f:-> Z;
  \mor        * q:-> Y g:-> *;

  \mor                       U   -> pb;
  \mor      :[bend left=10]  * x:-> X;
  \mor[swap]:[bend right=10] * y:-> Y;
\end{kodi}
\end{tcblisting}


%==[ SYNTAX: OBJECTS ]==========================================================

\newpage
\section{Syntax: objects}

% The first of the two macros that \CoDi\ offers is \lstinline|\obj|.
It is polymorphic and can draw both single objects and layouts.

\begin{lstlisting}
\obj@opt@ <object options> @/opt@{<math>};(@
  \marginpar{\scriptsize {\color{orange!80!black}Orange fragments} are optional.}@)
\obj@opt@ <layout options> @/opt@{<layout>};
\end{lstlisting}

Layouts are described using the customary \TeX\ tabular syntax.

\begin{lstlisting}
<layout>         := (@\itshape\underbar{<row> <row separator>}@)(@
  \marginpar{\scriptsize \underbar{Underlined fragments} can repeat one or more times.}@)
<row>            := <cell> (@\itshape\color{orange!80!black}\underbar{<cell separator> <cell>}@)
<row separator>  := \\ @opt@[<length>]@/opt@
<cell>           := @opt@|<object options>| @/opt@<math>
<cell separator> := & @opt@[<length>]@/opt@
\end{lstlisting}

The discretionary options syntax is analogous to standard \TikZ\ nodes and
matrices, respectively.

\begin{lstlisting}
<object options> := (@\itshape\color{orange!80!black}\underbar{[object keylist]}@) @opt@(<name>) at (<coordinate>)@/opt@
<layout options> := (@\itshape\color{orange!80!black}\underbar{[layout keylist]}@) @opt@(<name>) at (<coordinate>)@/opt@
\end{lstlisting}

\hfill$\therefore$\hfill\null

Nothing of the given syntax is specific to \CoDi.
In fact, \lstinline|\obj| can draw both single objects and layouts
by behaving like the standard \TikZ\ macros
\lstinline|\node| and \lstinline|\matrix| respectively.

Furthermore, layouts content is specified using the common \TeX\
tabular syntax.
The only catch is that row and column separators are always mandatory.

Here is a kitchen sink that includes custom spacing:

\begin{tcblisting}{snippet, trim}
\begin{codi}[square=3em]
\obj {
  A & B &[1em] C \\
  D & E &      F \\[-1em]
  G & H &      I \\
};
\end{codi}
\end{tcblisting}

Here is another one that includes custom options:

\begin{tcblisting}{snippet, trim}
\begin{codi}[square=3em]
\obj [red] {
  A & |[blue]| B & C \\
};
\end{codi}
\end{tcblisting}

A standard feature inherited from \TikZ\ worth a mention
is the ability to name a layout and refer to cells
by their row/column index pairs.

\begin{tcblisting}{snippet, trim}
\begin{codi}[square=3em]
\obj (M) { A & A \\ A & A \\ };
\node [draw=red,  shape=circle, minimum size=2em] at (M-1-2) {};
\node [draw=blue, shape=circle, minimum size=2em] at (M-2-1) {};
\end{codi}
\end{tcblisting}


%==[ SYNTAX: MORPHISMS ]========================================================

\newpage
\section{Syntax: morphisms}

% \begingroup\tcbset{trim/.default={3 and -1}}

The second and last macro that \koDi\ offers is \lstinline|\mor|.
It can draw single or chained morphisms.

\begin{lstlisting}
\mor@opt@ <chain options> @/opt@<object>(@
  \itshape\underbar{\textvisiblespace<morphism>\textvisiblespace<object>}@);(@
  \marginpar{\scriptsize Whitespace marked as \textvisiblespace\ is mandatory.}@)
\end{lstlisting}

Source and target objects are referred to by their name.
  
\begin{lstlisting}
<object>   := @nws@(<name>)@/nws@ (@
  \marginpar{\scriptsize {\color{blue!80!black}Blue fragments} can be either enclosed in the shown delimiters, or a \TeX\ group (not idiomatic), or simply devoid of whitespace.}@)
\end{lstlisting}

Morphisms consist of one or more optional labels and an arrow.
  
\begin{lstlisting}
<morphism> := @opt@<labels> : @/opt@<arrow>
<labels>   := @nws@"<math>"@/nws@ XOR (@\underbar{[{\itshape "<math>", <label keylist>}]}@) (@
  \marginpar{\scriptsize Alternatives are separated by $\vert$s.}@)
<arrow>    := @nws@[<arrow keylist>]@/nws@
\end{lstlisting}
% <labels>   := @nws@"<math>"@/nws@ XOR @nws@[<label keylist>]@/nws@ XOR (@\underbar{[{\itshape <label keylist>}]}@)

Global options can be given to both labels and arrows.

\begin{lstlisting}
<chain options> := [<label keylist>] @opt@: [<arrow keylist>]@/opt@
\end{lstlisting}

\hfill$\therefore$\hfill\null

These rules allow for a label syntax that sprouts gracefully
from the simplest to the most complex case.

\begin{tcblisting}{snippet, trim}
\begin{kodi}
\obj { A & B & C \\ F & E & D \\ };
\mor A -> B;
\mor B f:-> C;
\mor C \hat g:-> D;
\mor D "h i":-> E;
\mor E ["L", above]:-> F;
\mor F ["m", near start]["n", swap]["o", near end]:-> A;
\end{kodi}
\end{tcblisting}

The same holds for arrow syntax.

\begin{tcblisting}{snippet, trim}
\begin{kodi}
\obj { A & B & C \\ };
\mor A -> B;
\mor B [>-, dashed] C;
\end{kodi}
\end{tcblisting}

Global options can be used to minimize local ones
and keep the code terse and readable.

\begin{tcblisting}{snippet, trim}
\begin{kodi}
\obj { A & B & C \\ F & E & D \\ };
\mor [swap]:[bend left] B f:-> C g:>-> D h:>- E i:- B;
\mor :[bend right] E x:-> F y:>-> A z:>- B;
\mor [mid] B m:-> D;
\end{kodi}
\end{tcblisting}

\endgroup


%==[ GALLERY ]==================================================================

\newpage
\section{Gallery}

% % \makeatletter
% \newcommand\prev{\the\tikz@lastxsaved,\the\tikz@lastysaved}
% \makeatother

The remainder of the text is just commented examples.

\clearpage

\subsection{2-cells}

\begin{tcblisting}{gallery}
% From the LaTeX preamble:
%   \usepackage{tikz-cd}

% We use tikz-cd keys to quickly whip up an arrow style for a 2-morphism.
\pgfqkeys{/codi/arrows}{
  -2>/.style={
      /tikz/commutative diagrams/Rightarrow,
      /tikz/commutative diagrams/shorten=2pt,
  }
}

\begin{codi}
  \obj{ X & Y & Z \\ };
  \mor[swap]:[bend right] X f:-> Y m:-> Z;
  \mor      :[bend left]  X g:-> Y n:-> Z;
  \mor       f \alpha:-2> g;
  \mor[swap] m  \beta:-2> n;
\end{codi}
\end{tcblisting}

\subsection{Snake}

\begin{tcblisting}{gallery}
\begin{codi}[tetragonal]
  \obj{          & \ker a   & \ker b   & \ker c   &   \\
                 & A        & B        & C        & 0 \\
        |(0')| 0 & A'       & B'       & C'       &   \\
                 & \coker a & \coker b & \coker c &   \\ };

  \mor   (ker a) ->   (ker b) ->   (ker c);
  \mor (coker a) -> (coker b) -> (coker c);
  \mor       A  f :-> B  g :-> C -> 0;
  \mor 0' -> A' f':-> B' g':-> C';

  \mor[near start] (ker a) -> A a:-> A' -> (coker a);
  \mor[near start] (ker b) -> B b:-> B' -> (coker b);
  \mor[near start] (ker c) -> C c:-> C' -> (coker c);

  \draw[/codi/arrows/crossing over, ->, rounded corners, >=stealth]
    (ker c) -- ++( 0.6,0) -- ++(0,-1.55)
            -- ++(-3.2,0) -- ++(0,-1.45) -- (coker a);
\end{codi}
\end{tcblisting}

\clearpage

\subsection{The fourth associahedron}

\begin{tcblisting}{gallery,listing options={literate={×}{{\texttimes}}1 {ı}{{\i}}1}}
\begin{codi}
% From the LaTeX preamble:
%   \usepackage{newunicodechar}
%   \newunicodechar{ı}{\mathbf 1}
%   \newunicodechar{×}{\otimes}

  \foreach [count=\n] \o in {
      ((w×x)×y)×x,
      (w×(x×y))×x,
      w×((x×y)×x),
      w×(x×(y×x)),
      (w×x)×(y×x)
    } \obj (\n) at (72*\n:7em) {\o};

  \mor 1 "a_{w,x,y}×ı_z": -> 2
           "a_{w,x×y,z}": -> 3
         "ı_w×a_{x,y,z}": -> 4;
  \mor *   "a_{w×x,y,z}": -> 5
           "a_{w,x,y×z}": -> *;
\end{codi}
\end{tcblisting}

\clearpage

\subsection{Pullback \& pushout}

\begin{tcblisting}{gallery}
\begin{codi}[hexagonal]
  \obj{ |(pb)| A \times_Z B & B \\
               A            & Z \\ };
  \obj[above left=of pb] {Q};

  \mor[swap] pb p_1:-> A f:-> Z;
  \mor        * p_2:-> B g:-> *;

  \mor[swap]:[bend right] Q q_1:-> A;
  \mor      :[bend left]  * q_2:-> B;
  \mor [mid]:[dashed]     *   u:-> pb;
\end{codi}
\end{tcblisting}

\begin{tcblisting}{gallery}
\begin{codi}[hexagonal]
  \obj{ Z &                   B \\
        A & |(po)| A \sqcup_Z B \\ };
  \obj[below right=of po] {Q};

  \mor[swap] Z f:-> A i_1:-> po;
  \mor       * g:-> B i_2:-> *;

  \mor[swap]:[bend right]  A j_1:-> Q;
  \mor      :[bend left]   B j_2:-> *;
  \mor [mid]:[dashed]     po   u:-> *;
\end{codi}
\end{tcblisting}

\clearpage

\subsection{Complexes sequence}

\begin{tcblisting}{gallery}
\begin{codi}
  \obj (M) {   & \vdots  & \vdots  & \vdots  &   \\
             0 & A_{n+1} & B_{n+1} & C_{n+1} & 0 \\
             0 & A_{n}   & B_{n}   & C_{n}   & 0 \\
             0 & A_{n-1} & B_{n-1} & C_{n-1} & 0 \\
               & \vdots  & \vdots  & \vdots  &   \\ };

  \foreach \n/\row in {n+1/2, n/3, n-1/4}
    \mor (M-\row-1) -> (A_{\n}) "\alpha_{\n}":-> (B_{\n})
                                 "\beta_{\n}":-> (C_{\n}) -> (M-\row-5);

  \foreach \l/\col/\q in {A/2/, B/3/', C/4/''}
    \mor (M-1-\col) -> (\l_{n+1}) "\partial\q_{n+1}":-> (\l_{n})
                                  "\partial\q_{n}"  :-> (\l_{n-1}) -> (M-5-\col);
\end{codi}
\end{tcblisting}

\clearpage

\subsection{Braid}

% TODO: there's a bug here. If one does \begin{codi}[ l/.style={bend left}
% TODO: (note the space before l) then it's not recognized. Investigate.
\begin{tcblisting}{gallery}
\begin{codi}[l/.style={bend left}, r/.style={bend right} ]
  \obj [ hexagonal=horizontal side 6em angle 45, remove characters=H_\{q+\} ] {
    H_{q+2}(X)   & H_{q+2}(X,Y) & H_{q+1}(Y,Z) & H_{q}(Z)     \\
    H_{q+2}(Y)   & H_{q+2}(X,Z) & H_{q+1}(Y)   & H_{q+1}(X,Z) \\
    H_{q+2}(Y,Z) & H_{q+1}(Z)   & H_{q+1}(X)   &              \\
  };
  
  \mor :[blue]  2Y  -> 2X  l,-> 2XY   -> 1Y  -> 1X;
  \mor :[green] 2Y  -> 2YZ r,-> 1Z    -> 1Y  -> 1YZ l,-> Z;
  \mor :[cyan]  2X  -> 2XZ   -> 1Z  r,-> 1X  -> 1XZ   -> Z;
  \mor :[red]   2YZ -> 2XZ   -> 2XY l,-> 1YZ -> 1XZ;
\end{codi}
\end{tcblisting}

\clearpage

\subsection{Hammock}

\begingroup\catcode`~=12
\begin{tcblisting}{gallery}
\begin{codi}[x=4em, y=-3em, node distance=1 and 1,
    sim/.style={sloped, auto,
      edge node={node[every edge quotes][/velos/install quote
        handler,"\sim", anchor=south, outer sep=-.15em]}
    },
    ~>/.style={->, sim},
    <~/.style={<-, sim},
    ../.style={line width=.25ex, dash pattern=on 0sp off .75ex, line cap=round},
    remove characters=_\{\},
    expand=full,
  ]
    
  \foreach [count=\c] \col in {1, 2, 3, n}
  \foreach [count=\r] \row in {K_{\col}, C_{0\col}, \vdots, C_{m\col}, L_{\col}}
    \obj [name/.expanded={\ifnum\r=3 vdots\col\fi}] at (\c,\r) {\row};

  \obj  [left=of vdots1] {X};
  \obj [right=of vdotsn] {Y};

  \foreach \col in {1, 2, 3, n}
    \mor (K\col) ~> (C0\col) ~> (vdots\col) ~>  (Cm\col) ~> (L\col);
  
  \foreach \row in {K, C0, Cm, L} {
    \mor (\row1) -> (\row2) <~ (\row3) .. (\row n);
    \mor X <~ + -> Y;
  }
\end{codi}
\end{tcblisting}
\endgroup



\end{document}
