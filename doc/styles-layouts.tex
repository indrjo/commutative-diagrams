Layouts can be laid over regular grids:

\begin{lstlisting}
/codi/layouts/tetragonal=base <length> height <length>
  (@\hfill@) (default: base 4.5em height 2.8em)
\end{lstlisting}

\begin{lstlisting}
/codi/layouts/hexagonal=<direction> side <length> angle <angle>
  (@\hfill@) (default: horizontal side 4.5em angle 60)
\end{lstlisting}

When one of these keys is used the layout columns and rows will be
spaced and offset in order to reproduce the grids given by diagram styles.

\begin{tcblisting}{snippet, trim={2 and -1}}
\begin{codi}
  \obj [hexagonal=horizontal side 1.5em angle 60] {
    A & B &   \\
    C & D & E \\
    F & G &   \\
  };
\end{codi}
\end{tcblisting}

\begin{tcblisting}{snippet, trim={2 and -1}}
\begin{codi}
  \obj [hexagonal=vertical side 1.5em angle 60] {
    A & C & F \\
    B & D & G \\
      & E &   \\
  };
\end{codi}
\end{tcblisting}

Note that \emph{each row must have the same number of cells}%
\footnote{this is different from the behaviour of, say, tables}
or the spacing will be incorrect.

Note that these keys will be recognized only if you're using the tabular syntax of \lstinline|\obj|.
