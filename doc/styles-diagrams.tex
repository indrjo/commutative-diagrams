\tikzset{
  versor/.style={ultra thick, black, ->, >=latex, line cap=round},
  checkers/.style={fill=teal!15},
  measure/.style={decorate, decoration={brace, amplitude=4pt, aspect=#1}},
  measure/.default=0.5,
  label/.style={font=\scriptsize, outer sep=.5em, inner sep=0sp},
  dots/.style={fill=teal!50},
  label style/.style={label, outer sep=.2em, font={\ttfamily\tiny}},
  anchor style/.style={inner sep=0sp, outer sep=0sp, anchor=center},
  pin style/.style={pin distance=.75em, pin edge={thin, black, shorten <=3pt}, label style}
}

\begin{marginfigure}[0cm]
  \begin{subfigure}{\marginparwidth}
    \begin{tikzpicture}[codi, tetragonal=base {0.2\marginparwidth} height {0.2*1.618\marginparwidth}]
      % drawing area
      \clip (-2.5,-2.5) rectangle (2.5,2.5);
      % checkers and dots
      \foreach \i in {-3, ..., 2} \foreach \j in {-3, ..., 2} {
        \pgfmathparse{int(1+abs(\i)+abs(\j))}
        \ifodd\pgfmathresult \fill [checkers] (\i,\j) rectangle +(1,1);\fi
        \pgfmathparse{ \i == sign(\i) && \j == sign(\j) && abs(\i) != -abs(\j) }
        \fill [fill/.expanded={\ifnum\pgfmathresult=1 red\else teal!50\fi}]
          (\i,\j) circle (2pt);
      }
      % base circle  
      \fill [red, opacity=0.1] (1,1) -- (-1,1) -- (-1,-1) -- (1,-1) -- cycle;
      % width marker
      \draw
        [measure, decoration=mirror, yshift=-3pt]
        (0,0) -- (1,0)
        node [label style, midway, below, yshift=-3pt] {base};
      % height marker
      \draw
        [measure=0.7, xshift=-3pt]
        (0,0) -- (0,1)
        node [label style, pos=0.7, left, xshift=-3pt] {height};
      % origin
      \coordinate (O) at (0,0);
      \node at (O) [anchor style, pin={[pin style]-135:{O}}] {};
      % neighbours
      \node [anchor style, pin={[pin style]-45:{above=of O}}, above=of O] {};
      \node [anchor style, pin={[pin style]60:{above left=of O}}, above left=of O] {};
      \node [anchor style, pin={[pin style]90:{left=1 of O}}, left=1 of O] {};
      \node [anchor style, pin={[pin style]120:{below right=2 of O}}, below right=2 of O] {};
      \node [anchor style, pin={[pin style]-45:{below left=1 and 2 of O}}, below left=1 and 2 of O] {};
      % versors
      \draw [versor] (0,0) -- (1,0);
      \draw [versor] (0,0) -- (0,1);
    \end{tikzpicture}
    \subcaption*{Tetragonal}\bigskip
  \end{subfigure}
  \begin{subfigure}{\linewidth}
    \begin{tikzpicture}[codi, hexagonal=horizontal side {0.4*\marginparwidth} angle 60]
      % drawing area
      \clip (-2.5,-2.5) rectangle (2.5,2.5);
      % checkers and dots
      \foreach \i in {-4, ..., 2} \foreach \j in {-3, ..., 2} {
        \pgfmathparse{int(abs(\i)+abs(\j))}
        \ifodd\pgfmathresult\else \fill [checkers] (\i,\j) -- +(1,1) -- +(2,0);\fi
        \pgfmathsetmacro\dotsize{isodd(\pgfmathresult)?1:2}
        \fill [fill/.expanded={\ifnum\pgfmathresult=2 red\else teal!50\fi}]
          (\i,\j) circle (\dotsize pt);
      }
      % base circle  
      \fill [red, opacity=0.1]
        (2,0) -- (0,2) -- (-2,0) -- (0,-2) -- cycle;
      % side marker
      \draw [measure, decoration=mirror, yshift=-3pt]
        (0,0) -- (2,0)
        node [label style, midway, below, yshift=-3pt] {side};
      % angle marker
      \draw (0.66*1.7cm, 0)
        arc (0:30:0.66*1.7cm)
        node [label style, above right] {angle}
        arc (30:60:0.66*1.7cm);
      % origin
      \coordinate (O) at (0,0);
      \node at (O) [anchor style, pin={[pin style]-135:{O}}] {};
      % neighbours
      \node [anchor style, pin={[pin style]-45:{above=of O}}, above=of O] {};
      \node [anchor style, pin={[pin style]60:{above left=of O}}, above left=of O] {};
      \node [anchor style, pin={[pin style]90:{left=1 of O}}, left=1 of O] {};
      \node [anchor style, pin={[pin style]120:{below right=2 of O}}, below right=2 of O] {};
      \node [anchor style, pin={[pin style]-45:{below left=1 and 2 of O}}, below left=1 and 2 of O] {};
      % versors
      \draw [versor] (0,0) -- (1,0);
      \draw [versor] (0,0) -- (0,1);
    \end{tikzpicture}
    \subcaption*{Hexagonal (horizontal)}\bigskip
  \end{subfigure}
  \begin{subfigure}{\linewidth}
    \begin{tikzpicture}[codi, hexagonal=vertical side {0.2*\marginparwidth*2/sqrt(3)} angle 60]
      % drawing area
      \clip (-2.5,-2.5) rectangle (2.5,2.5);
      % checkers and dots
      \foreach \i in {-3, ..., 2} \foreach \j in {-4, ..., 3} {
        \pgfmathparse{int(1+abs(\i)+abs(\j))}
        \ifodd\pgfmathresult \fill [checkers] (\i,\j) -- +(1,1) -- +(0,2);\fi
        \pgfmathsetmacro\dotsize{isodd(\pgfmathresult)?2:1}
        \fill [fill/.expanded={\ifnum\pgfmathresult=3 red\else teal!50\fi}]
          (\i,\j) circle (\dotsize pt);
      }
      % base circle  
      \fill [red, opacity=0.1] (2,0) -- (0,2) -- (-2,0) -- (0,-2) -- cycle;
      % side marker
      \draw [measure=0.3, xshift=-3pt]
        (0,0) -- (0,2) node [label style, pos=0.3, left, xshift=-3pt]  {side};
      % angle marker
      \draw 
        ([shift=(30:0.66*0.982cm)]0,0)
        arc (30:60:0.66*0.982cm)
        node [label style, above right] {angle}
        arc (60:90:0.66*0.982cm);
      % origin
      \coordinate (O) at (0,0);
      % neighbours
      % versors
      \draw [versor] (0,0) -- (1,0);
      \draw [versor] (0,0) -- (0,1);
    \end{tikzpicture}
    \subcaption*{Hexagonal (vertical)}\bigskip
  \end{subfigure}
  %% \caption*{Diagram grid examples}
\end{marginfigure}

Diagrams can be laid over regular grids:

\begin{lstlisting}
/codi/diagrams/tetragonal=base <length> height <length>
  (@\hfill@) (default: base 4.5em height 2.8em)
\end{lstlisting}

\begin{lstlisting}
/codi/diagrams/hexagonal=<direction> side <length> angle <angle>
  (@\hfill@) (default: horizontal side 4.5em angle 60)
\end{lstlisting}

When one of these keys is used
\begin{itemize}[noitemsep]
  \item the versors of the \NiceURL{coordinate system}{http://texdoc.net/texmf-dist/doc/generic/pgf/pgfmanual.pdf\#page=357} are changed,
  \item the \NiceURL{node positioning}{http://texdoc.net/texmf-dist/doc/generic/pgf/pgfmanual.pdf\#page=229} is set up to lay them on grid,
  \item and the corresponding key will be applied to all layouts.
\end{itemize}

The pictures show the key parameters, versors, and a unitary grid.

This setup allows you to mix coordinates and
relative positioning keys to arrange objects.

As usual, relative positioning keys can accept two components, a radius,
or nothing at all (which defaults to a certain radius).

When using a radius (or defaulting to $1$)
the tetragonal grid uses Manhattan distance to
lay objects along concentric rectangles.

When using a radius (or defaulting to $2$)
the hexagonal grid%
\footnote{which in truth is built upon a tetragonal grid}
uses Chebyshev distance to
lay objects along concentric rhombi.

To clarify, a few relative positioning keys are drawn along
with red zones displaying the default radii around the origins.
