\begin{lstlisting}[style=metacode]
/codi/arrows/crossing over
/codi/arrows/crossing over/clearance=<length> (@\hfill@) (default: 0.5ex)
/codi/arrows/crossing over/color=<color> (@\hfill@) (default: white)
\end{lstlisting}

This key a provides the configurable illusion of an arrow passing
over a \emph{previously drawn} one.

\begin{tcblisting}{snippet, trim={3 and -1}}
\begin{codi}[tetragonal=base 4.5em height 1em]
\obj { A & B \\ D & C \\};
\mor A -> C;
\mor :[crossing over] D -> B;
\end{codi}
\end{tcblisting}

%%%%%%%%%%%%%%%%%%%%%%%%%%%%%%%%%%%%%%%%%%%%%%%%%%%%%%%%%%%%%%%%%%%%%%%%%%%%%%%%

\begin{lstlisting}[style=metacode]
/codi/arrows/slide=<length>
\end{lstlisting}

This key slides an arrow backward (negative) and forward (positive) along its direction of the given length.

\begin{tcblisting}{snippet, trim={3 and -1}}
\begin{codi}[tetragonal=base 4.5em height 1em]
\obj { A & B \\ C & D \\ E & F \\ };
\mor :[slide=-.3em, red] A -> B;
\mor C -> D;
\mor :[slide=+.3em, blue] E -> F;
\end{codi}
\end{tcblisting}

%%%%%%%%%%%%%%%%%%%%%%%%%%%%%%%%%%%%%%%%%%%%%%%%%%%%%%%%%%%%%%%%%%%%%%%%%%%%%%%%

\begin{lstlisting}[style=metacode]
/codi/arrows/shove=<length>
\end{lstlisting}

This key shoves an arrow to the left (negative) and to the right (positive) with respect to its direction of the given length.

\begin{tcblisting}{snippet, trim={3 and -1}}
\begin{codi}
\obj { A & B \\ };
\mor :[shove=-.3em, red] A -> B;
\mor A -> B;
\mor :[shove=+.3em, blue] A -> B;
\end{codi}
\end{tcblisting}

%%%%%%%%%%%%%%%%%%%%%%%%%%%%%%%%%%%%%%%%%%%%%%%%%%%%%%%%%%%%%%%%%%%%%%%%%%%%%%%%

\hfill$\therefore$\hfill\null

%%%%%%%%%%%%%%%%%%%%%%%%%%%%%%%%%%%%%%%%%%%%%%%%%%%%%%%%%%%%%%%%%%%%%%%%%%%%%%%%

\CoDi\ is currently missing a base arrow style library.

You can define your own styles adding them to \lstinline|/codi/arrows/|.

If you're familiar with {\ttfamily\small tikz-cd}, you can import its arrow styles to use them with \CoDi\ as follows in you preamble:
\begin{lstlisting}
\usepackage{tikz-cd}
\pgfqkeys{/codi}{
    every arrow/.append style={
        /ektropi/add=/tikz/commutative diagrams
    }
}
\end{lstlisting}

Then, you can use them seamlessly:

\begin{tcblisting}{snippet, trim={2 and -1}}
\begin{codi}
\obj{ X & Y \\ & Z \\ };
\mor X f:->,hook Y g:-> Z;
\mor X f:dashrightarrow Z;
\end{codi}
\end{tcblisting}
