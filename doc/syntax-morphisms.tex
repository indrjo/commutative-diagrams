\begingroup\tcbset{trim/.default={3 and -1}}

The second and last macro that \koDi\ offers is \lstinline|\mor|.
It can draw single or chained morphisms.

\begin{lstlisting}
\mor@opt@ <chain options> @/opt@<object>(@
  \itshape\underbar{\textvisiblespace<morphism>\textvisiblespace<object>}@);(@
  \marginpar{\scriptsize Whitespace marked as \textvisiblespace\ is mandatory.}@)
\end{lstlisting}

Source and target objects are referred to by their name.
  
\begin{lstlisting}
<object>   := @nws@(<name>)@/nws@ (@
  \marginpar{\scriptsize {\color{blue!80!black}Blue fragments} can be either enclosed in the shown delimiters, or a \TeX\ group (not idiomatic), or simply devoid of whitespace.}@)
\end{lstlisting}

Morphisms consist of one or more optional labels and an arrow.
  
\begin{lstlisting}
<morphism> := @opt@<labels> : @/opt@<arrow>
<labels>   := @nws@"<math>"@/nws@ XOR (@\underbar{[{\itshape "<math>", <label keylist>}]}@) (@
  \marginpar{\scriptsize Alternatives are separated by $\vert$s.}@)
<arrow>    := @nws@[<arrow keylist>]@/nws@
\end{lstlisting}
% <labels>   := @nws@"<math>"@/nws@ XOR @nws@[<label keylist>]@/nws@ XOR (@\underbar{[{\itshape <label keylist>}]}@)

Global options can be given to both labels and arrows.

\begin{lstlisting}
<chain options> := [<label keylist>] @opt@: [<arrow keylist>]@/opt@
\end{lstlisting}

\hfill$\therefore$\hfill\null

These rules allow for a label syntax that sprouts gracefully
from the simplest to the most complex case.

\begin{tcblisting}{snippet, trim}
\begin{kodi}
\obj { A & B & C \\ F & E & D \\ };
\mor A -> B;
\mor B f:-> C;
\mor C \hat g:-> D;
\mor D "h i":-> E;
\mor E ["L", above]:-> F;
\mor F ["m", near start]["n", swap]["o", near end]:-> A;
\end{kodi}
\end{tcblisting}

The same holds for arrow syntax.

\begin{tcblisting}{snippet, trim}
\begin{kodi}
\obj { A & B & C \\ };
\mor A -> B;
\mor B [>-, dashed] C;
\end{kodi}
\end{tcblisting}

Global options can be used to minimize local ones
and keep the code terse and readable.

\begin{tcblisting}{snippet, trim}
\begin{kodi}
\obj { A & B & C \\ F & E & D \\ };
\mor [swap]:[bend left] B f:-> C g:>-> D h:>- E i:- B;
\mor :[bend right] E x:-> F y:>-> A z:>- B;
\mor [mid] B m:-> D;
\end{kodi}
\end{tcblisting}

\endgroup
