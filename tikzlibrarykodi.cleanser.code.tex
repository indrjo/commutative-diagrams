% tikzlibrarykodi.cleanser.code.tex
%
% Copyright 2015 by Paolo Brasolin <paolo.brasolin@gmail.com>
%
% This program is free software: you can redistribute it and/or modify
% it under the terms of the GNU General Public License as published by
% the Free Software Foundation, either version 3 of the License, or
% (at your option) any later version.
%
% This program is distributed in the hope that it will be useful,
% but WITHOUT ANY WARRANTY; without even the implied warranty of
% MERCHANTABILITY or FITNESS FOR A PARTICULAR PURPOSE.  See the
% GNU General Public License for more details.
% 
% You should have received a copy of the GNU General Public License
% along with this program.  If not, see <http://www.gnu.org/licenses/>.

% ######################################################################## SOF #

% The cleanser is based upon a circuit between three TikZ keys:
%   input --(1)-> process --(2)-> output

% First we define three possible behaviors of the expansion to be performed on
% the input before processing.  These correspond to three mutually exclusive
% different wirings for (1).  By default the no-expansion route is wired.
\pgfqkeys{/cleanser/expand}{.is choice,
  none/.style={/cleanser/input/.style={/cleanser/process={##1}}},
  once/.style={/cleanser/input/.style={/cleanser/process/.expand once={##1}}},
  full/.style={/cleanser/input/.style={/cleanser/process/.expanded={##1}}},
  none,} % default behaviour

\newif\ifcleanseroverwrites
\pgfqkeys{/cleanser}{
  overwrite/.is if=cleanseroverwrites,
  overwrite=false,} % default behaviour

% Then we wire (2) using the Plain TeX sanitizer and the TikZ gizmo to handle
% console output that we're defining later.
\pgfqkeys{/cleanser/process}{.code={
  \kDSanitize#1\into tmp\GO
  \pgfkeysalso{/cleanser/log={\Meaning = "\tmp"}}
  \pgfkeysalso{/cleanser/output/.expand once=\tmp}},}

% Here is the handler for console output.
\newif\ifcleanserisverbose
\pgfqkeys{/cleanser}{
  verbose/.is if=cleanserisverbose,
  verbose=true,
  log/.code={
    \ifcleanserisverbose
      \let~\space
      \newlinechar=`\^
      \message{kD.cleanser ~~~ > #1^}
    \fi},
  /cleanser/output/.forward to=/cleanser/dispatcher,
  dispatcher/.code={
    \ifx\tikz@fig@name\empty
      \pgfqkeys{/tikz}{name=#1}
    \else
      \ifcleanseroverwrites
        \pgfqkeys{/tikz}{name=#1}
      \fi
    \fi
  },
}

% Here is the main reason for all this: a style to make a node self-naming.
% Again, it's just a matter of wiring a simple sircuit.
\pgfqkeys{/tikz/self naming node}{.style={
  /tikz/node contents/.forward to=/cleanser/input,
},}

% No more TikZ code is needed.  What follows s just the Plain TeX sanitizer.

\def\kDMeanCut#1:->#2\into#3\GO%
  {\kDCSDef{#3}{#2}}

\def\kDMeaningOf#1\into#2\GO%
  {\def\kDFoo{#1}\edef\kDBar{\meaning\kDFoo}%
   \expandafter\kDMeanCut\kDBar\into#2\GO}

\def\kDSanitize#1\into#2\GO%
  {\kDMeaningOf#1 \into Meaning\GO% I put an extra space for easy parsing
   \expandafter\edef\csname#2\endcsname% and catch it as the empty exception
     {\ifx\Meaning\space\else\expandafter\kDSieveString\Meaning\GO\fi}}

% Maybe a switch to preserve/erase spaces would be useful here
\def\kDSieveString#1 #2\GO%
  {\kDSieveWord#1\GO\ifx\relax#2\else\space\kDSieveString#2\GO\fi}

\def\kDSieveWord#1#2\GO%
  {\kDSieveCharacter#1\GO\ifx\relax#2\else\kDSieveWord#2\GO\fi}

% Spaces after number to avoid stray \relax
\def\kDSieveCharacter#1\GO%
  {\ifnum`#1=36 \else% $ dollar
   \ifnum`#1=40 \else% ( left round brace
   \ifnum`#1=41 \else% ) right round brace
   \ifnum`#1=44 \else% , comma
   \ifnum`#1=46 \else% . dot
   \ifnum`#1=58 \else% : colon
   \ifnum`#1=92 \else% \ backslash
   #1\fi\fi\fi\fi\fi\fi\fi}

% ######################################################################## EOF #
