\documentclass[12pt]{scrbook}
\usepackage[british]{babel}
% \hyphenation{Fortran hy-phen-ation}

\usepackage{libertine}
\usepackage{libertinust1math}
\usepackage[ttdefault=true]{AnonymousPro}

\usepackage[utf8]{inputenc}
\usepackage[T1]{fontenc}

% \def\id{{\mathbf 1}}
% \DeclareUnicodeCharacter{0131}{{\noexpand\mathbf R}}
% \DeclareUnicodeCharacter{00D7}{{\otimes}}
% \catcode`ı=\active \letı\id
% \catcode`×=\active \let×\otimes
% \usepackage{newunicodechar}
% \newunicodechar{ı}{\mathbf 1}
% \newunicodechar{×}{\otimes}

\setkomafont{disposition}{\rmfamily\scshape}

\usepackage{geometry}

\geometry{
  a4paper,
  portrait,
  marginparwidth=4.25cm,
  marginparsep=.75cm,
  % showframe,
  width=11cm,
  hmarginratio=10:25,
  vmarginratio=20:30,
%  footskip=.5in,
}
%\savegeometry{main}

%\usepackage{multicol}
%\usepackage{tabularx}
%\usepackage{verbatimbox}
% \usepackage{wrapfig}

\usepackage{parskip}

\usepackage{mathtools}
\usepackage{amsfonts}
\usepackage{amssymb}


\DeclareMathOperator{\coker}{coker}

\usepackage{float}
% \usepackage{framed}
% \usepackage{lipsum}
% \usepackage{enumitem}
% \usepackage{pdfpages}

% \usepackage{caption}
\usepackage{sidenotes}

% \DeclareCaptionStyle{sidecaption}%
% {font=footnotesize,labelfont=sc}

% \usepackage{csquotes}

\usepackage{graphicx}
% \usepackage[makeroom]{cancel}

\usepackage{etoolbox}

% \usepackage{todonotes}
% \usepackage{marginnote}

% \usepackage[lastpage,user]{zref}
% \usepackage[hidelinks]{hyperref}

\usepackage{xcolor}
\usepackage{tikz}
% \usetikzlibrary{cd}
% \usetikzlibrary{quotes}
% \pgfkeys{/handlers/first char syntax=false}

\usetikzlibrary{kodi}
\usetikzlibrary{positioning}

% \usepackage[all]{xy}
% \usepackage{pstricks,pst-node}


\usepackage{xparse}

\usepackage[
  headwidth=textwithmarginpar,
  footwidth=textwithmarginpar,
]{scrlayer-scrpage}
%\clearpairofpagestyles

% \lofoot{}
% \cofoot{}
% \rofoot{}
% \lohead{}
% \cohead{}
% \rohead{\thepage}

%\usepackage{multicol}

\usepackage{microtype}

\usepackage{listings}

% http://tex.stackexchange.com/a/336331/82186
\makeatletter
\lst@Key{lastline}\relax{\ifnumcomp{#1}{<}{0}{%
  \let\mylst@file\lst@intname\sbox0{\lstinputlisting{\mylst@file}}%
  \def\lst@lastline{\the\numexpr#1+\value{lstnumber}-1\relax}}%
  {\def\lst@lastline{#1\relax}}}
\makeatother



\usepackage{showexpl}
\makeatletter
\lst@Key{postset}\relax{\def\SX@postset{#1}}
\newcommand\SX@postset{}
\renewcommand*\SX@resultInput{%
  \ifx\SX@graphicname\@empty
    \begingroup
      \MakePercentComment\makeatother\catcode`\^^M=5\relax
      \SX@@preset\SX@preset
      \if@SX@rangeaccept
       \let\SX@tempa=\SX@input
      \else
       \let\SX@tempa=\input
      \fi
      \if\SX@scaled ?%
        \let\SX@tempb=\@firstofone
      \else
        \if\SX@scaled !%
          \def\SX@tempb##1{\resizebox{\SX@width}{!}{##1}}%
        \else
          \def\SX@tempb##1{\scalebox{\SX@scaled}{##1}}%
        \fi
      \fi
      \SX@tempb{\SX@tempa{\SX@codefile}}\SX@postset\par
    \endgroup
  \else
    \expandafter\includegraphics\expandafter[\SX@graphicparam]%
      {\SX@graphicname}%
  \fi
}
\makeatother

\lstdefinelanguage{TikZ}
{morekeywords={for},
sensitive=false,
morecomment=[l]{//},
morecomment=[s]{/*}{*/},
morestring=[b]",
}

\usepackage{color,soul}
\definecolor{darkblue}{rgb}{0,0,0.5}
\setulcolor{darkblue}

\lstset{
  language=[LaTeX]TeX,
  basicstyle=\ttfamily\lst@ifdisplaystyle\scriptsize\fi,
  backgroundcolor=\color{teal!5},
  % keywordstyle=*\color{blue},
  % identifierstyle=\color{orange}\bfseries,
  % morekeywords={\path},
  moretexcs={
    \starttext,\stoptext,\usemodule,
    \tikzpicture,\endtikzpicture,
    \tikzexternalize,
    \starttikzpicture,\stoptikzpicture,
    \usetikzlibrary,
    \kodi,\endkodi,
    \startkodi,\stopkodi,
    \lay,\obj,\mor,
    \bye
  },
  texcsstyle=*\textbf,
  % morestring=[b]",
  commentstyle=\itshape,
  frame=none,
  % extendedchars=false,
  % inputencoding=utf8,
  escapeinside={(@}{@)},
  moredelim=**[is][\color{orange!80!black}]{@opt@}{@/opt@},
  moredelim=**[s][\itshape]{<}{>},
  literate={:=}{{$\equiv$}}1 {~}{{\textvisiblespace}}1,
}

% \makeatletter
% \newcount\ublvl\ublvl=0
% \newcount\ubdpt\ubdpt=0
% \newdimen\ubgap\ubgap=.2em
% \def\underbra#1{\underline {\sbox \tw@ {\global\advance\ubdpt1\advance\ublvl1#1}\dp \tw@ \dimexpr\ubgap*(\ubdpt-\ublvl-1)\relax \box \tw@ }\ifnum\ublvl=0\ubdpt=0\fi}
% \makeatother

\lstset{explpreset={
  wide,
  basicstyle=\ttfamily\scriptsize,
  pos=o,
  width=\marginparwidth,
  hsep=\marginparsep,
  rframe={},
  preset={\centering\tikzpicture[kodi]},
  postset={\endtikzpicture}
}}

\usepackage{lipsum}
\usepackage{caption}
\usepackage{subcaption}

\captionsetup[marginfigure]{
  textfont=bf,
  justification=centering
}
\captionsetup[subfigure]{
  textfont=normalfont,
  singlelinecheck=off,
  justification=centering
}

\usepackage{hologo}
\def\ConTeXt{\hologo{ConTeXt}}
\def\koDi{{\scshape koDi}}
\def\TikZ{{\scshape TikZ}}

\usepackage{tcolorbox}
\tcbuselibrary{listingsutf8}
\tcbset{listing utf8=latin1}

\begin{document}

%===[ TITLE PAGE ]==============================================================

\thispagestyle{empty}
\noindent
\resizebox{\linewidth}{!}{\scshape koDi}\\[.6em]
\resizebox{\linewidth}{!}{\scshape kommutative Diagramme für \TeX}\\[1.6em]
\resizebox{\linewidth}{!}{\scshape enchiridion}\par
\vfill\hfill
\marginpar{
  \resizebox{\linewidth}{!}{\scshape\color{red} unreleased}\\[.6em]
  \resizebox{\linewidth}{!}{\scshape v0.0.0}\\
  \resizebox{\linewidth}{!}{\scshape \today}
}
\newpage

%===[ FOREWORD ]================================================================

% \begin{adjustwidth}{.45\textwidth}{.45\textwidth-\marginparwidth-\marginparsep}
\noindent\koDi\ is a \TikZ\ library. Its purpose
is drawing commutative diagrams.
It is designed precisely to overcome
the shortcomings of traditional ones.
The syntax is minimalistic and intelligible.\par
\hfill{\itshape Paolo al-Imkānī Brasolin}
% \end{adjustwidth}

\newpage

%===[ PRELIMINARIES ]===========================================================

\section{Preliminaries}

\TikZ\ is the only requirement of \koDi.  This ensures compatibility with
most \TeX\ flavours.  Here are minimal working examples for the main dialects:

\begin{figure}[H]
  \begin{adjustwidth}{0sp}{-\marginparwidth-\marginparsep}
    \begin{subfigure}{\marginparwidth}
      \caption*{\TeX}
      \begin{lstlisting}

\input kodi

\kodi
  % diagram here
\endkodi
\bye
      \end{lstlisting}
    \end{subfigure}
    \hfill
    \begin{subfigure}{\marginparwidth}
      \caption*{\ConTeXt\ module}
      \begin{lstlisting}

\usemodule[kodi]
\starttext
\startkodi
  % diagram here
\stopkodi
\stoptext
      \end{lstlisting}
    \end{subfigure}
    \hfill
    \begin{subfigure}{\marginparwidth}
      \caption*{\LaTeX\ package}
      \begin{lstlisting}
\documentclass{article}
\usepackage{kodi}
\begin{document}
\begin{kodi}
  % diagram here
\end{kodi}
\end{document}
      \end{lstlisting}
    \end{subfigure}
    \par
    \begin{subfigure}{\marginparwidth}
      \caption*{\TeX\ (\TikZ\ library)}
      \begin{lstlisting}

\input tikz
\usetikzlibrary kodi

\tikzpicture[kodi]
  % diagram here
\endtikzpicture
\bye
      \end{lstlisting}
    \end{subfigure}
    \hfill
    \begin{subfigure}{\marginparwidth}
      \caption*{\ConTeXt\ (\TikZ\ library)}
      \begin{lstlisting}

\usemodule[tikz]
\usetikzlibrary[kodi]
\starttext
\starttikzpicture[kodi]
  % diagram here
\stoptikzpicture
\stoptext
      \end{lstlisting}
    \end{subfigure}
    \hfill
    \begin{subfigure}{\marginparwidth}
      \caption*{\LaTeX\ (\TikZ\ library)}
      \begin{lstlisting}
\documentclass{article}
\usepackage{tikz}
\usetikzlibrary{kodi}
\begin{document}
\begin{tikzpicture}[kodi]
  % diagram here
\end{tikzpicture}
\end{document}
      \end{lstlisting}
    \end{subfigure}
  \end{adjustwidth}
\end{figure}

A useful \TikZ\ feature exclusive to \LaTeX\ is externalization.
A small expedient is necessary to use it with \koDi.
\begin{marginfigure}[-2em]
  \caption*{\TikZ\ externalization}
  \begin{lstlisting}
\documentclass{article}
\usepackage{tikz}
\usetikzlibrary{kodi}
\usetikzlibrary{external}
\tikzexternalize
  [prefix=tikzpicfolder/]
\begin{document}
\begin{tikzpicture}[kodi]
  % diagram here
\end{tikzpicture}
\end{document}
  \end{lstlisting}
\end{marginfigure}

\tcbset{kodi snippet/.style={
  size=tight,
  colback=white,
  colframe=white,
  if odd page={
    listing side text,
    lefthand width=\textwidth,
    righthand width=\marginparwidth,
    halign lower=center,
  }{
    text side listing,
    righthand width=\textwidth,
    lefthand width=\marginparwidth,
    halign upper=center,
  },
  toggle enlargement,
  grow to right by=\marginparsep+\marginparwidth,
  sidebyside gap=\marginparsep,
  listing options={
    firstline=2,
    lastline=-1
  }
}}

\tcbset{xy snippet/.style={
  size=tight,
  colback=white,
  colframe=white,
  if odd page={
    listing side text,
    lefthand width=\textwidth,
    righthand width=\marginparwidth,
    halign lower=center,
  }{
    text side listing,
    righthand width=\textwidth,
    lefthand width=\marginparwidth,
    halign upper=center,
  },
  toggle enlargement,
  grow to right by=\marginparsep+\marginparwidth,
  sidebyside gap=\marginparsep,
}}

%===[ QUICK TOUR ]==============================================================

\newpage

\def\nilstrut{\rule{0sp}{0sp}}

\section{Quick tour}

Objects are typeset using the \lstinline!\obj! macro.
%
\begin{tcblisting}{kodi snippet}
\nilstrut\smash{\tikzpicture[baseline=(current bounding box.center),kodi]
\obj {X};
\endtikzpicture}
\end{tcblisting}
%
Almost every diagram is laid along a regular grid,
so the customary tabular syntax of \TeX\ is recognized.
%
\begin{tcblisting}{kodi snippet}
\nilstrut\smash{\tikzpicture[baseline=(current bounding box.center),kodi]
\obj {
  A & B \\
  C & D \\
};
\endtikzpicture}
\end{tcblisting}
%
\koDi\ objects are self-aware and clever enough to name themselves
so you can comfortably refer to them.
%
\begin{tcblisting}{kodi snippet}
\nilstrut\smash{\tikzpicture[baseline=(current bounding box.center),kodi]
\obj {\lim F};
\draw (lim F) circle (4ex);
\endtikzpicture}
\end{tcblisting}
%
Morphisms are typeset using the \lstinline!\mor! macro.
%
\begin{tcblisting}{kodi snippet}
\nilstrut\smash{\tikzpicture[baseline=(current bounding box.center),kodi]
\obj { A & B \\ };
\mor A f:-> B;
\endtikzpicture}
\end{tcblisting}
%
Commutative diagrams exist to study composition and commutation
so naturally \koDi\ allows for the chaining of morphisms and
the gluing of chains.
%
\begin{tcblisting}{kodi snippet}
\nilstrut\smash{\tikzpicture[baseline=(current bounding box.center),kodi]
\obj { A & B \\ C & D \\ };
\mor A -> B -> D;
\mor * -> C -> *;
\endtikzpicture}
\end{tcblisting}
%
These are the only two macros defined by \koDi.

There are more features so please continue reading if this got your attention.

%===[ COMPETITION ]=============================================================

% \newpage
% 
% \section{Competitors}
% 
% \begin{tcblisting}{xy snippet, listing options={linerange=1-5}}
% \xymatrix{
 % U \ar@/_/[ddr]_y \ar[dr] \ar@/^/[drr]^x \\
  % & X \times_Z Y \ar[d]^q \ar[r]_p
                 % & X \ar[d]_f            \\
  % & Y \ar[r]^g & Z                       }
% \end{tcblisting}
% 
% yo yo yo 
% 
% 
% \begin{tcblisting}{xy snippet, listing options={linerange=1-5}}
% \[ \psset{arrows=->, arrowinset=0.25, linewidth=0.6pt, nodesep=3pt, labelsep=2pt, rowsep=1.2cm}
% \begin{psmatrix}
  % (X, d) & (X_1 ,d_1 )\\%
   % & (X_2 ,d_2)
% %%%
 % \ncline{1,1}{1,2}\naput{T_1} \ncline{1,1}{2,2}\nbput{T_2 }
 % \ncline{1,2}{2,2}\naput[npos=0.45]{T}
% \end{psmatrix}
% \]
% \end{tcblisting}
% 
% 
% 
% wot wot
% 
% \begin{tcblisting}{kodi snippet, listing options={linerange=2-13}}
% \tikzpicture[baseline=(current bounding box.center),kodi]
% \obj {
  % U \\
    % & |(P)| X \times_Z Y & X \\
    % &                  Y & Z \\
% };
% 
% \mor[swap] P p:-> X f:-> Z;
% \mor       * q:-> Y g:-> *;
% 
% \mor                    U   -> P;
% \mor      :[bend left]  * x:-> X;
% \mor[swap]:[bend right] * y:-> Y;
% \endtikzpicture
% \end{tcblisting}

%===[  ]========================================================================

\newpage

\section{EBNF specification}

% \begin{lstlisting}
% \obj@opt@ [<object opts>] (<name>) at (<coordinate>) @/opt@{<math>};
% \end{lstlisting}
%
% \begin{lstlisting}
% \obj@opt@ [<object opts>] (<name>) at (<coordinate>) @/opt@{<math>};
% \end{lstlisting}
%

\begin{lstlisting}
\lay@opt@ [<layout opts>] (<name>) at (<coordinate>) @/opt@{<table>};

<table> := (@\underbar{\itshape<row>}@)
<row> := <cell> (@\itshape\color{orange!80!black}\underbar{<col sep> <cell>}@) <row sep>
<col sep> := & @opt@[<length>]@/opt@
<row sep> := \\ @opt@[<length>]@/opt@
\end{lstlisting}


\begin{lstlisting}
\mor@opt@ [<label opts>] : [<arrow opts>]@/opt@ <object>(@\itshape\underbar{\textvisiblespace<morphism>\textvisiblespace<object>}@);
<morphism> := <labels> : <arrow>
\end{lstlisting}


\begin{lstlisting}
<one or more foo> := @rep@<foo>@/rep@
<zero or more foo> := @opt@@rep@<foo>@/rep@@/opt@
<zero or one foo> := @opt@<foo>@/opt@
<row> := <cell> @opt@@rep@<col sep> <cell>@/rep@@/opt@ <row sep>
<col sep> := & @opt@[<length>]@/opt@
<row sep> := \\ @opt@[<length>]@/opt@
\end{lstlisting}







%%%%%%%%%%%%%%%%%%%%%%%%%%%%%%%%%%%%%%%%%%%%%%%%%%%%%%%%%%%%%%%%%%%%%%%%%%%%%%%%

% \lipsum[1]
% 
% \begin{LTXexample}
% \lay [golden=2em] { A & B & C \\ D & E & F \\ };
% \mor A [bend right, ->] B [bend left, ->] C
  % -> D [bend right, ->] E [bend left, ->] F;
% \end{LTXexample}
% 
% \lipsum[2]
% 
% \begin{LTXexample}
% \lay { A & B \\};
% \mor A -> B;
% \end{LTXexample}
% 
% \lipsum[3]
% 
% 
% 
% % \show\ker
% 
% \begin{LTXexample}[pos=t]
% \begin{tikzpicture}[kodi]
% 
% \lay[golden]{
           % & \ker a   & \ker b   & \ker c   &   \\
           % & A        & B        & C        & 0 \\
  % |(0')| 0 & A'       & B'       & C'       &   \\
           % & \coker a & \coker b & \coker c &   \\
% };
% 
% % horizontal chains
% \mor   (ker a) ->   (ker b) ->   (ker c);
% \mor (coker a) -> (coker b) -> (coker c);
% \mor       A  f :-> B  g :-> C -> 0;
% \mor 0' -> A' f':-> B' g':-> C';
% 
% % vertical chains
% \mor[near start] (ker a) -> A a:-> A' -> (coker a);
% \mor[near start] (ker b) -> B b:-> B' -> (coker b);
% \mor[near start] (ker c) -> C c:-> C' -> (coker c);
% 
% % the snake
% \coordinate (tail) at ($(ker b)!9/4!(ker c)$);
% \coordinate (head) at ($(coker b)!9/4!(coker a)$);
% \coordinate (belly) at ($(B)!9/16!(B')$);
% \draw[/kD/arrows/crossing over, ->, rounded corners]
  % (ker c) -- (tail) -- (tail|-belly)
          % -- (belly-|head) -- (head) -- (coker a);
% 
% \end{tikzpicture}
% \end{LTXexample}
% 
% \newpage
% 
% % \begin{LTXexample}[pos=t]
% % \begin{tikzpicture}[kodi]
% % 
% % \foreach [count=\n] \o in
    % % {((w×x)×y)×x,
     % % (w×(x×y))×x,
     % % w×((x×y)×x),
     % % w×(x×(y×x)),
     % % (w×x)×(y×x)}
  % % \obj (\n) at ({72*\n:9em}) {\o};
% % 
% % \mor 1 "a_{w,x,y}×ı_z": -> 2
         % % "a_{w,x×y,z}": -> 3
       % % "ı_w×a_{x,y,z}": -> 4;
% % \mor *   "a_{w×x,y,z}": -> 5
         % % "a_{w,x,y×z}": -> *;
% % 
% % \end{tikzpicture}
% % \end{LTXexample}
% 
% % \startbuffer[example-pullback]%%%%%%%%%%%%%%%%%%%%%%%%%%%%%%%%%%%%%%%%%%%%%%%%%%
% % \lay[comb]{
  % % |(P)| A \times_Z B & B \\
        % % A            & Z \\
% % };
% % 
% % \obj at ($(Z)!2!(P)$) {Q};
% % 
% % \mor[swap] P p_1:-> A f:-> Z;
% % \mor       * P_2:-> B g:-> *;
% % \mor[swap]:[bend right] Q q_1:-> A;
% % \mor      :[bend left]  * q_2:-> B;
% % \mor [mid]:[dashed]     *   u:-> P;
% % \stopbuffer%%%%%%%%%%%%%%%%%%%%%%%%%%%%%%%%%%%%%%%%%%%%%%%%%%%%%%%%%%%%%%%%%%%%%
% % 
% % \startbuffer[example-chain]%%%%%%%%%%%%%%%%%%%%%%%%%%%%%%%%%%%%%%%%%%%%%%%%%%%%%
% % \pgfkeys{/katharizo/expand=full}
% % 
% % \foreach [count=\m] \a in {A,B,C}
  % % \foreach [count=\n] \i in {n-2,n-1,n,n+1,n+2}
    % % \obj at ({5em*\n,-3em*\m}) {\a_{\i}};
% % 
% % \mor (C_{n+1}) -> (B_{n-1});
% % \stopbuffer%%%%%%%%%%%%%%%%%%%%%%%%%%%%%%%%%%%%%%%%%%%%%%%%%%%%%%%%%%%%%%%%%%%%%
% % 
% % \startbuffer[example-ref]%%%%%%%%%%%%%%%%%%%%%%%%%%%%%%%%%%%%%%%%%%%%%%%%%%%%%%%
% % \lay[square]{A&B&C&D\\E&F&G&H\\I&L&M&N\\};
% % 
% % \mor:                     B -> C;
% % 
% % % * src = prv first src
% % \mor:[draw=none]          B -> C;
% % \mor:[bend right, green]  * -> D;
% % 
% % % * tar = prv last tar
% % \mor:[draw=none]          B -> C;
% % \mor:[bend right, blue]   A -> *;
% % 
% % \mor:[draw=none]          B -> C;
% % \mor:[bend left, red]     A -> * -> D;
% % 
% % \mor:                     F -> G;
% % 
% % % + src = prv last tar
% % \mor:[draw=none]          F -> G;
% % \mor:[bend right, green]  + -> H;
% % 
% % % + tar = prv first src
% % \mor:[draw=none]          F -> G;
% % \mor:[bend right, blue]   E -> +;
% % 
% % \mor:[draw=none]          F -> G;
% % \mor:[bend left, red]     E -> + -> H;
% % 
% % % so, basically, * is the opposite of +
% % \mor:      M -> L;
% % \mor:[red] I <- * <- N;
% % \stopbuffer%%%%%%%%%%%%%%%%%%%%%%%%%%%%%%%%%%%%%%%%%%%%%%%%%%%%%%%%%%%%%%%%%%%%%

\end{document}
